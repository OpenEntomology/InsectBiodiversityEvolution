\documentclass[letterpaper, 11pt]{article}
\usepackage{comment} % enables the use of multi-line comments (\ifx \fi) 
\usepackage{fullpage} % changes the margin
\usepackage{fancyhdr} % for footer
\usepackage[UKenglish]{isodate}% http://ctan.org/pkg/isodate for date format
\usepackage[letterpaper,margin=1in]{geometry}
\usepackage{epigrafica}%changes default font to epigrafica
\newcommand{\latinword}[1]{\texttt{\itshape #1}}%

\def\labelitemi{--}

\pagestyle{fancy}
\renewcommand{\headrulewidth}{0pt}
\lhead{}
\chead{}
\rhead{}
\lfoot{ENT 432 (Fall 2016) - Penn State}
\cfoot{}
\rfoot{\thepage}
\renewcommand{\footrulewidth}{0.4pt}

\title{Unit 2 - Basic arthropod morphology}
\author{Open Entomology Project}
\begin{document}
\cleanlookdateon %removed ordinal date
\maketitle
\thispagestyle{fancy}
\section*{Introduction}
The external features of an arthropod, \textit{i.e.}, the phenotypes we can easily observe through a microscope, are the primary source of characters we use to diagnose arthropods to higher-level taxonomic groupings. These phenotypes also inform us about natural history and inspire hypotheses of evolutionary relationships. Given the extraordinary diversity of this lineage, however, which certainly contains more than 1,000,000 species, the terminology entomologists use to refer to anatomical structures (and even to describe their form) is large, convoluted, and rich in homonyms and synonyms. In this lab we will observe specimens from across the phylogeny of Arthropoda, with an emphasis on Insecta. The goal is to familiarize yourself with the anatomical concepts relevant to most diagnostic keys and phylogenetic studies and to learn the various terms applied to these structures. Terms you should be comfortable with are differentiated by \latinword{font}.

\section*{Materials}
\begin{itemize}
	\item {specimens (provided)}
	\item {fine forceps, probes (provided)}
	\item {sorting tray, watch glasses, gloves, safety glasses, glycerine (provided)}
	\item {lab notebook (\textit{not} provided!)}
\end{itemize}

\subsection*{Taxa}
\begin{enumerate}
	\item {spider (Araneae)}
	\item {harvestmen (Opiliones)}
	\item {woodlouse (Isopoda)}
	\item {mayfly (Ephemeroptera)}
	\item {grasshopper (Orthoptera)}
	\item {backswimmer (Heteroptera: Notonectidae)}
	\item {planthopper (Hemiptera)}
	\item {ground beetle (Coleoptera: Carabidae)}
	\item {ground beetle bleached with H\textsubscript{2}O\textsubscript{2} (Coleoptera: Carabidae)}
	\item {weevil (Coleoptera: Curculionidae)}
	\item {moth (Lepidoptera)}
	\item {robber fly (Diptera: Asilidae)}
	\item {hover fly (Diptera: Syrphidae)}
	\item {wasp (Hymenoptera)}
	\item {sawfly (Hymenoptera)}
	\item {springtail (Collembola)}
	\item {antlion (Neuroptera: Myrmeleontidae)}
\end{enumerate}

\section*{Safety}
We will be working with sharp tools. Wear your personal protective gear at all times. Specimens are to be returned to their vials after lab, and glycerine will be collected for proper disposal or reuse.

\section*{Methods}
Working with a partner, organize your space, specimens, tools, and microscope. Use your probe and forceps to manipulate the specimen, and don't be afraid to pull parts off or otherwise dissect the specimens. Use the provided morphology handout as a guide.

\section{Classify your specimens}
Take a minute or two to sort the specimens into groups. \\

\hangindent2em{}\textbf{Question 2-1:} How many groups did you make, and what body parts provided the evidence or characters to justify your groups?

\section{Integumentary basics}
\latinword{Segments} are metameric subdivisions of the body or appendages. Some researchers use ``segment'' only for those metameric subdivisions that meet certain criteria, for example if they each have intrinsic musculature. Using your forceps and probe, try to expand and contract your arthropod specimens. Separate two \latinword{flagellomeres} (subdivisions of the apical segment of the \latinword{antenna}, the \latinword{flagellum}) and then the \latinword{pedicel} (the medial antennal segment) from the \latinword{scape} (proximal segment of the antenna) on one of the beetles.\\

\hangindent2em\textbf{Question 2-2:} Which body parts are segmented? How are segments separated from each other? Are all metameric subdivisions of an insects \textit{true} segments? Do you see any differences?\\

\noindent{}Try to separate two segments on the \latinword{abdomen} of a specimen. Then pull apart the abdomen of a specimen where the abdomen is protracted (\textit{e.g}., in the bleached carabid beetle), and observe the difference in stiffness between the \latinword{conjunctiva} and a \latinword{sclerite}. \\

\hangindent2em\textbf{Question 2-3:} What components of the segment define the segment's boundaries? Conjunctiva is much weaker than sclerites; why do insects and other arthropods need conjunctiva?\\

\noindent{}Select a specimen to examine under the microscope; a beetle would be good for this purpose. You should see hair-like cuticular modifications, \latinword{setae} (seta, sing.), which usually serve mechanosensory function. Sometimes these are modified as scales---\textit{i.e.}, they are flattened---which can also serve to present certain colors. If you look carefully you may also find other cuticular modifications: sensilla (\latinword{sensillum}, sing.), which usually look like pits or holes, and pores.\\

\hangindent2em\textbf{Question 2-4:} Can you locate setae on each specimen and explain why they are located on these body regions and what function they serve? Could you find other cuticular specializations that might have sensory function (gustation, olfaction)? Where are these sensors located?\\

\section{Head}
Tagmata (\latinword{tagma}, sing.) are body regions composed of several segments that function together for certain, specialized purposes. The first tagma we'll look at is the \latinword{head}. Spend some time examining the heads of your specimens, both of insects and of non-insects. Note the number and location of appendages and look for evidence of segmentation. \\

\hangindent2em\textbf{Question 2-5:} How many segments do you think comprise this tagma? Do non-insect arthropods have the same head configuration as Insecta? What is the major function of the head? \\

\noindent{}The antenna is the most obvious metameric appendage of the head. As we have seen earlier, the flagellum of the insect antenna is not musculated (only the first two sclerites, the scape and the pedicel, which are true \latinword{appendage segments}, have muscle attachments). Some basal hexapods (like the springtail in the petri dish) and non-hexapod pancrustaceans have fully musculated antennae. \\

\hangindent2em\textbf{Question 2-6:} How many antennae do insects have? And the spider and woodlouse? Some arthropods do not have antennae, while others have highly reduced antennae. How do these arthropods, like the harvestman (Opiliones), compensate for the lack of this useful structure? Why don't insects need a musculated flagellum?\\

\hangindent2em\textbf{Question 2-7:} You observe here different antenna phenotypes. Why are antennae so diverse in shape?\\

\noindent{}Now focus again on the beetle specimens; find and examine other head appendages. Some questions to think about while you're making your observations: Are the \latinword{mandibles} metameric? Can you find any appendages on the head besides the antenna that are metameric? Are they always paired? How many non-antennal, metameric head appendages could you differentiate? Are they exclusively composed of ring-like sclerites? Expand your observations beyond the beetles and see if you can understand the mouthparts of your arachnids.\\

\hangindent2em\textbf{Question 2-8:} What might be the function of the \latinword{maxilla} and the \latinword{labium}? Are these structures (plus the mandibles) present on all your specimens?\\

\hangindent2em\textbf{Question 2-9:} For these arthropods,  describe your hypotheses about the primary foods for each of your specimens (what do they eat and why?): spider, mayfly, ground beetle, moth, fly, springtail (Can you even see this one's mouthparts?!)\\

\noindent{}Also note the orientation of mouthparts. They can project anteriorly (\latinword{prognathous}), ventrally (\latinword{hypognathous}), or posteriorly (\latinword{opisthognathous}). These different orientations reflect modifications of the \latinword{cranium}. Find an example of each in your set of specimens. \\

\hangindent2em\textbf{Question 2-10:} What might be the reason or function of these different mouthpart positions?\\

\noindent{}Eyes are used for vision and light detection. Locate them on your specimens and think about how they are differentiated from the rest of the body (cuticle).\\

\hangindent2em\textbf{Question 2-11:} How many eyes do the Hymenoptera specimens have? What might be the function of a \latinword{compound eye} \textit{vs}. an \latinword{ocellus} (-i)?\\

\section{Thorax}
Posterior to the head is usually another tagma called the  \latinword{thorax}. As with the head, examine each specimen for evidence of segmentation in this region. Questions to think about: Does each specimen have a thorax? How many segments do you think comprise this tagma? How many appendages to you see, and where are they attached?\\

\hangindent2em\textbf{Question 2-12:} What is the primary function of this tagma? \\

\hangindent2em\textbf{Question 2-13:} In your insect specimens the first segment does not bear wing. Why? Does this body region serve other functions?\\

\noindent{}In your hexapod specimens, find sclerites you think represent the \latinword{notum}, \latinword{sternum}, and \latinword{pleuron}. Note that the surfaces of the nota of the hymenopteran specimens have many \latinword{furrows}, separating convex regions.\\

\hangindent2em\textbf{Question 2-14:} What might be the function of these convex regions? Are there internal structures that correspond to these cuticular modifications?\\

\hangindent2em\textbf{Question 2-14:} Can you locate the notum on the second thoracic segment of the carabid beetle? Why it is reduced?\\

\noindent{}Try to locate the \latinword{fore wing} and the \latinword{hind wing} on all your insect specimens. Note that the fore wing of beetles is modified into an \latinword{elytron} (-a) that is not participating much in flight but rather serves as protection for the hind wing (among other possible functions you'll learn about later). Think about the characteristics and parts of the wings that allow them to function for flight. \\

\hangindent2em\textbf{Question 2-15:} What might be the function of \latinword{wing veins}? Is it possible to group wing veins based on their orientation (relative to the anterior wing margin) on the antlion specimen? And on the hymenopteran and fly specimens? Longitudinal and \latinword{crossveins} cannot be differentiated in the latter two taxa. Why? \\

\hangindent2em\textbf{Question 2-16:} Given that the hind wing is so much larger than the elytron in your beetles, how can it fit under it?\\

\noindent{}Locate the following veins in different specimens: \latinword{costa} (C), \latinword{subcosta} (Sc), \latinword{radius} (R), \latinword{cubitus} (Cu), \latinword{medius} (M), \latinword{anal} (A). Look also for structures that aren't necessarily veins (fold lines, for example) but which otherwise aid in the function of the wing.\\

\hangindent2em\textbf{Question 2-17:} Do you think it is possible to homologize these veins across insects? \\

\noindent{}In some basal winged insect taxa the fore and hind wing are used asynchronously during the flight. Hymenoptera, some Heteroptera, and Lepidoptera use the fore and hind wings simultaneously (\textit{i.e}., they function as a single wing). \\ 

\hangindent2em\textbf{Question 2-18:} Can you find any anatomical structures that link or otherwise connect the two wings together on the Hymenopteran specimens?\\

\noindent{}Locate the following leg segments in your insects: \latinword{coxa}, \latinword{trochanter}, \latinword{femur}, \latinword{tibia}, \latinword{tarsus}, \latinword{tarsomere}, \latinword{pretarsal claw}. Now try to locate these segments on the arachnid specimens. Questions to think about: Does each leg have the same number of segments? Does each thoracic segment have a leg? Do the legs vary in their morphology between segments? What about anteriorly to posteriorly?\\


\hangindent2em\textbf{Question 2-19:} Can you tell something about the biology of the following specimens based on their leg phenotypes: ground beetle, backswimmer, grasshopper, weevil.\\

\noindent{}How do insects ``stick'' to certain substrates that appear to be too smooth to climb? Think about a fly crawling on a window. Look for adaptations you would consider to be \latinword{attachment devices} on the legs.

\section{Abdomen}
As with the tagmata above, see if you can find evidence for a third tagma posterior to the thorax---the \latinword{abdomen}---and for how many segments compose the structure.\\

\hangindent2em\textbf{Question 2-20:} How does the abdomen articulate with the thorax?\\

\noindent{}Look for appendages on the abdomen. \\

\hangindent2em\textbf{Question 2-21:} Could you separate male and female insects based on the tip of their abdomen? What might be the functions of the anatomical structures at the tip of the abdomen, for example the \latinword{cerci} (cercus, sing.) or the \latinword{ovipositor}? \\

\hangindent2em\textbf{Question 2-22:} The terminalia of the abdomen is key in species identification for numerous insect groups. Why? \\

\noindent{}In most insects there are two sets of \latinword{spiracles} on the thorax. Try to locate them in your winged arthropods, and then count the number of spiracles on the abdomen. Note that the second spiracle on the thorax is often difficult to find because it is located at the wing base. \\

\hangindent2em\textbf{Question 2-23:} How many spiracles could you find in the hymenopteran specimens? Why is the number different from other winged insects?\\

\hangindent2em\textbf{Question 2-24:} Why is the abdomen generally softer and more flexible than the other tagmata? Why do we pin insects through the thorax?

\end{document}
