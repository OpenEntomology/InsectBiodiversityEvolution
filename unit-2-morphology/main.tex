\documentclass[letterpaper, 11pt]{article}
\usepackage{comment} % enables the use of multi-line comments (\ifx \fi) 
\usepackage{fullpage} % changes the margin
\usepackage{fancyhdr} % for footer
\usepackage[UKenglish]{isodate}% http://ctan.org/pkg/isodate for date format
\usepackage[letterpaper,margin=1in]{geometry}

\def\labelitemi{--}

\pagestyle{fancy}
\renewcommand{\headrulewidth}{0pt}
\lhead{}
\chead{}
\rhead{}
\lfoot{ENT 432 (Fall 2016) - Penn State}
\cfoot{}
\rfoot{\thepage}
\renewcommand{\footrulewidth}{0.4pt}

\title{Unit 2 - Basic arthropod morphology}
\author{Andrew R. Deans and Istv\'an Mik\'o}
\begin{document}
\cleanlookdateon %removed ordinal date
\maketitle
\thispagestyle{fancy}
\section*{Introduction}
The external features of an arthropod, \textit{i.e.}, the phenotypes we can easily observe through a microscope, are the primary source of characters we use to diagnose arthropods to higher-level taxonomic groupings. These phenotypes also inform us about natural history and inspire hypotheses of evolutionary relationships. Given the extraordinary diversity of this lineage, however, which certainly contains more than 1,000,000 species, the terminology entomologists use to refer to anatomical structures (and even to describe their form) is large, convoluted, and rich in homonyms and synonyms. In this lab we will observe specimens from across the phylogeny of Arthropoda, with an emphasis on Insecta. The goal is to familiarize yourself with the anatomical concepts relevant to most diagnostic keys and phylogenetic studies and to learn the various terms applied to these structures. Terms in bold are ones you should be comfortable with.

\section*{Materials}
\begin{itemize}
	\item {specimens (provided)}
	\item {fine forceps, probes (provided)}
	\item {sorting tray, watch glasses, gloves, safety glasses, glycerine (provided)}
	\item {pencil/paper for sketches}
\end{itemize}

\subsection*{Taxa}
\begin{enumerate}
	\item {spider (Araneae)}
	\item {harvestmen (Opiliones)}
	\item {woodlouse (Isopoda)}
	\item {mayfly (Ephemeroptera)}
	\item {grasshopper (Orthoptera)}
	\item {backswimmer (Heteroptera: Notonectidae)}
	\item {planthopper (Hemiptera)}
	\item {ground beetle (Coleoptera: Carabidae)}
	\item {ground beetle bleached with H\textsubscript{2}O\textsubscript{2} (Coleoptera: Carabidae)}
	\item {weevil (Coleoptera: Curculionidae)}
	\item {moth (Lepidoptera)}
	\item {robber fly (Diptera: Asilidae)}
	\item {hover fly (Diptera: Syrphidae)}
	\item {wasp (Hymenoptera)}
	\item {sawfly (Hymenoptera)}
	\item {springtail (Collembola)}
	\item {antlion (Neuroptera: Myrmeleontidae)}
\end{enumerate}

\section*{Safety}
We will be working with sharp tools. Wear your personal protective gear at all times. Specimens are to be returned to their vials after lab, and glycerine will be collected for proper disposal or reuse.

\section*{Methods}
Working with a partner, organize your space, specimens, tools, and microscope. Use your probe and forceps to manipulate the specimen, and don't be afraid to pull parts off or otherwise dissect the specimens. Use the provided morphology handout as a guide.

\section*{Classify your specimens}
Take a minute or two to sort the specimens into groups. How many groups did you make, and what body parts provided the evidence or characters to justify your groups?
\vspace{30 mm}

\section*{Integumentary basics}
\textbf{Segments} are ``metameric'' subdivisions. Which body parts are segmented? How are segments separated from each other? Are all metameric subdivisions of an insects true segments? Try to separate two \textbf{flagellomeres} (subdivisions of the apical segment of the \textbf{antenna}, the \textbf{flagellum}) and then the \textbf{pedicel} (the medial antennal segment) from the \textbf{scape} (proximal segment of the antenna)on one of the beetles. Do you see any differences?
\vspace{30 mm}

\noindent{}Try to separate two segments on the \textbf{abdomen} of a specimen. What components of the segment define the segment's boundaries? Pull apart the abdomen of some specimens, where it is protracted (\textit{e.g}., in the bleached carabid beetle), and observe the difference in stiffness between the \textbf{conjunctiva} and a sclerite. Conjunctivae are much weaker than sclerites; why do insects and other arthropods need conjunctivae?
\vspace{30 mm}

\noindent{}The larger, hair-like cuticular modifications, \textbf{setae} (seta, sing.), usually have mechanosensory function. Can you locate setae on each specimen and explain why they are located on these body regions? Can you find seta-like cuticular \textbf{appendages}, focusing on the weevil and the moth, that perhaps are not sensory? Could you find other cuticular specializations that might have sensory function (gustation, olfaction)? Where are these sensors located?
\vspace{30 mm}

\section*{Head}
How many segments do you think comprise this \textbf{tagma}? Do non-insect arthropods have the same \textbf{head} configuration as Insecta? What is the major function of the head? 
\vspace{30 mm}

\noindent{}The antenna is the most obvious metameric appendage of the head. How many antennae do insects have? And the woodlouse?
\vspace{30 mm}

\noindent{}As we have seen earlier, the flagellum of the insect antenna is not musculated (only the first two sclerites, the scape and the pedicel, which are true \textbf{appendage segments}, have muscle attachments). Some basal hexapods (like the springtail in the petri dish) and non-hexapod pancrustaceans have fully musculated antennae. Why don’t insects need a musculated flagellum? Some arthropods do not have antennae, while others have highly reduced antennae. How do these arthropods, like the harvestman (Opiliones), compensate for the lack of this useful structure?
\vspace{30 mm}


\noindent{}You observe here different antenna phenotypes. Why are antennae so diverse in shape?
\vspace{30 mm}

\noindent{}Now focus again on the large carabid beetles. Excluding the antenna, what appendages do you observe on the head? What might be their function? Are the \textbf{mandibles} metameric? Can you find any appendages on the head besides the antenna that are metameric? Are they always paired? How many non-antennal, metameric head appendages could you differentiate? Are they exclusively composed of ringlike sclerites? What might be the function of the \textbf{maxilla} and the \textbf{labium}?
\vspace{30 mm}

\noindent{}Can you locate the mouthparts on other specimens? Do you think these structures are absent from any of these organisms?
\vspace{30 mm}

\noindent{}Now look at the two arachnid specimens. Do the \textbf{chelicerae} and \textbf{pedipalps} of the spider match anything you see on the insects?
\vspace{30 mm}

\noindent{}In the list below describe your hypotheses about the primary foods for each of your specimens (what do they eat and why?):
\begin{enumerate}
\item spider 
\item mayfly 
\item carabid beetle
\item moth 
\item fly
\item springtail
\end{enumerate}

\noindent{}Eyes are used for vision and light detection. Can you locate them on the specimens? How can you differentiate an eye from the rest of the body (cuticle)? How many eyes do the Hymenoptera specimens have? What might be the function of a \textbf{compound eye} vs. an \textbf{ocellus} (-i)?
\vspace{30 mm}

\noindent{}Also note the orientation of mouthparts. Do they project anteriorly (\textbf{prognathous}), ventrally (\textbf{hypognathous}), or posteriorly (\textbf{opisthognathous})? These different orientations reflect modifications of the \textbf{cranium}. What might be the reason or function of these different mouthpart positions?
\vspace{30 mm}

\section*{Thorax}
Does each specimen have a \textbf{thorax}? How many segments do you think comprise this \textbf{tagma}? What is the primary function of this tagma? 
\vspace{30 mm}

\noindent{}In your insect specimens the first segment does not bear wing. Why? Does this body region serve other functions?
\vspace{30 mm}

\noindent{}Find sclerites you think represent the \textbf{notum}, \textbf{sternum}, and \textbf{pleuron}. Are they easy to delimit? 
\vspace{30 mm}

\noindent{}The surfaces of the nota of the hymenopteran specimens have many \textbf{furrows}, separating convex regions. What might be the function of these regions? Are there any internal structures corresponding to these external cuticular modifications?
\vspace{30 mm}

\noindent{}Can you locate the notum on the second thoracic segment of the carabid beetle? Why it is reduced?
\vspace{30 mm}

\noindent{}Can you locate the \textbf{fore wing} and the \textbf{hind wing} on all insect specimens? The fore wing of the beetles is modified into an \textbf{elytron} (-a) that is not participating in flight but rather serves as protection for the fore wing (among other possible functions you'll learn about later). It is much smaller in area than the hind wing. How can it cover the hind wing (focus on carabid beetles). 
\vspace{30 mm}

\noindent{}Can you find the hind wing on the weevil? You see wing veins not only on the hind wing but also on the elytra on the carabid specimen. What might be the function of the \textbf{wing vein}? Do you have any idea about the ontogenetic development of the wing (\textit{i.e}., do you have a hypothesis for the origins of these structures)? 
\vspace{30 mm}

\noindent{}Is it possible to group wing veins based on their orientation (relative to the anterior wing margin) on the antlion specimen? And on the hymenopteran and fly specimens? Longitudinal and \textbf{crossveins} can not be differentiated in the latter two taxa. What might be the reason of that? 
\vspace{30 mm}

\noindent{}Please locate the following veins in different specimens: \textbf{costa} (C), \textbf{subcosta} (Sc), \textbf{radius} (R), \textbf{cubitus} (Cu), \textbf{medius} (M), \textbf{anal} (A). Do you think it is possible to homologize these veins across insects? 
\vspace{30 mm}


\noindent{}What are the characteristics of the wing that makes it feasible for flight? Do you think \textbf{wing cells} are conjunctivae or sclerites?
\vspace{30 mm}


\noindent{}In some basal winged insect taxa the fore and hind wing are used asynchronously during the flight. Hymenoptera, some Heteroptera, and Lepidoptera use the fore and hind wings simultaneously (\textit{i.e}., they function as a single wing). Can you find any anatomical structure that link or connect the two wings together on the Hymenopteran specimens? 
\vspace{30 mm}

\noindent{}Besides the wing veins, do you see any other longitudinal lines on the wings of the wasp? What might be the function of the fold and flexion lines?
\vspace{30 mm}

\noindent{}Locate the following leg segments: \textbf{coxa}, \textbf{trochanter}, \textbf{femur}, \textbf{tibia}, \textbf{tarsus}, \textbf{tarsomere}, \textbf{pretarsal claw}. try to locate these segments on the arachnid specimens, as well. Does each thoracic segment have a leg? Do the legs vary in their morphology between segments? 
\vspace{30 mm}

\noindent{}Can you tell something about the biology of the following specimens based on their leg phenotypes:
\begin{enumerate}
\item ground beetle
\item backswimmer
\item grasshopper
\item weevil
\end{enumerate}

\noindent{}How do insects ``stick'' to certain substrates that appear to be too smooth to climb. Think about a fly crawling on a window. Do you see any \textbf{attachment devices} on the legs?

\section*{Abdomen}
Does each specimen have an \textbf{abdomen}? How does it articulate with the thorax?
\vspace{25 mm}

\noindent{}Do you see any appendages on the abdomen? Could you separate male and female insects based on the tip of their abdomen? What might be the functions of the anatomical structures at the tip of the abdomen, for example the \textbf{cerci} (cercus, sing.) or the \textbf{ovipositor}? 
\vspace{30 mm}

\noindent{}The terminalia of the abdomen is key in species identification for numerous insect groups. Why? 
\vspace{30 mm}

\noindent{}Can you locate any \textbf{spiracles} on the abdomen? The thorax has 2 spiracles in most insects. Could you locate them? The second spiracle is hard to find since it is located at the wing base. How many spiracles could you find in hymenopterans? Could you provide an explanation of the presence of a 3rd spiracle of the second tagma in the wasp?
\vspace{30 mm}

\noindent{}Why is the abdomen generally softer and more flexible than the other tagmata? Why do we pin insects through the thorax?

\end{document}
