\documentclass[letterpaper, 11pt]{article}
\usepackage{comment} % enables the use of multi-line comments (\ifx \fi) 
\usepackage{fullpage} % changes the margin
\usepackage{fancyhdr} % for footer
\usepackage[UKenglish]{isodate}% http://ctan.org/pkg/isodate for date format
\usepackage[letterpaper,margin=1in]{geometry}
\def\labelitemi{--}

\pagestyle{fancy}
\renewcommand{\headrulewidth}{0pt}

\lhead{}
\chead{}
\rhead{}
\lfoot{ENT 432 (Fall 2016) - Penn State}
\cfoot{}
\rfoot{\thepage}
\renewcommand{\footrulewidth}{0.4pt}
\title{Unit 16 - Natural History Collections}
\author{Andrew R. Deans}
\begin{document}
\cleanlookdateon %removed ordinal date
\maketitle
\thispagestyle{fancy}
\section*{Introduction}
The collection building exercise is designed to contribute substantial, robust data to the larger scientific enterprise. Your specimens could be used in future research on the ecology and evolution of insects, and it's even possible that one or more of your specimens represent new species. This lecture focuses on the types of data you collected, how those data are availed, the types of research that are facilitated by natural history collections, and other uses of these resources.

\section*{Bigger picture}

\noindent{}How many kinds of data can you list that are found in the collection you built as a class?\\

\noindent{}Given the data types listed in the previous question, can you describe three research projects that could be built around them? What is the most compelling use of natural history collections that you have seen, either in lecture or that you discovered on your own?\\

\noindent{}What conditions are required to effectively share biodiversity data?\\

\noindent{}Can you describe two ethical issues regarding specimen collection? \\

\noindent{}Regardless of your domain in entomology, or biology for that matter, you should be interacting with a natural history collection. Minimally this includes the deposition of vouchers. How will your voucher specimens look, and how would you reference them in a publication?\\


\end{document}
