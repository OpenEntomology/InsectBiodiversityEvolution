\documentclass[11pt,letterpaper]{article}
\usepackage{comment} % enables the use of multi-line comments (\ifx \fi) 
%
\usepackage[letterpaper,margin=0.75in]{geometry}
\usepackage{fullpage} % changes the margin
\usepackage{fancyhdr} % for footer
\usepackage[UKenglish]{isodate}% http://ctan.org/pkg/isodate for date format
\usepackage{epigrafica}%changes default font to epigrafica
\usepackage{hyperref}%for URLs
\usepackage[]{natbib}
\newcommand*{\doi}[1]{\href{http://dx.doi.org/#1}{doi: #1}}% links DOI

\pagestyle{fancy}
\renewcommand{\headrulewidth}{0pt}

\lhead{}
\chead{}
\rhead{}
\lfoot{ENT 432 (Fall 2017) - Penn State}
\cfoot{}
\rfoot{\thepage}
\renewcommand{\footrulewidth}{0.4pt}
\title{Paper discussions}
\author{Open Entomology Project}

\begin{document}
\cleanlookdateon %removed ordinal date
\maketitle
\thispagestyle{fancy}
\section*{Introduction}
This course is made up of 15 units, representing important topics in insect evolution and biodiversity. Each unit opens with a paper discussion that should help us gain a better idea of the research and history of how we understand insect evolution. These papers go beyond the phylogeny of the orders, providing information on the hypotheses, experiments, and analyses that brought us to the current ideas about hexapod evolution. For each paper there will be an assigned \textbf{leader} (or leaders), and the remaining students will be \textbf{discussants}. 

\paragraph*{If you are a leader ...} you should read the assigned paper carefully well ahead of time and prepare to facilitate a discussion about it. Why was this paper chosen, or why do you think it is relevant to the current theme? Talk to you instructors! Do some background reading---who are the authors, what is their research domain, what is the historical context of this paper, how does it compare to or inform current research, \textit{etc}.---and put this paper into context. Think about the main discussion points, and remember that \textit{PowerPoint and other slide programs are not allowed}! These are chalk talks, and they should last at least 20 minutes.

\paragraph*{If you are a discussant ...} you are not leading the discussion, but you \textit{are} contributing to its development. Read the paper carefully and think about why it's important. Are there elements you don't understand? What surprised you? How would carry the research forward, building on its results?\\

\noindent{}Check \url{http://canvas.psu.edu} for discussion dates and PDFs of each reading.

%\clearpage
%\section*{Epilogue}
%This handout is part of an open curriculum. Original files are available free for anyone to download, copy, modify, and improve at the Open Entomology GitHub repository \citep{ENT432}, which also provides a mechanism for reporting problems and other feedback:\\
%\url{https://github.com/OpenEntomology/InsectBiodiversityEvolution/issues}

% adding bibliography here
%\bibliographystyle{myplainnat}
%\bibliography{bib}

\end{document}
