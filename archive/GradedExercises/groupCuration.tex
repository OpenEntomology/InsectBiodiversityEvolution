\documentclass[letterpaper, 11pt]{article}
\usepackage{comment} % enables the use of multi-line comments (\ifx \fi) 
\usepackage{fullpage} % changes the margin
\usepackage{fancyhdr} % for footer
\usepackage[UKenglish]{isodate}% http://ctan.org/pkg/isodate for date format
\usepackage[letterpaper,margin=1in]{geometry}
\def\labelitemi{--}

\pagestyle{fancy}
\renewcommand{\headrulewidth}{0pt}

\lhead{}
\chead{}
\rhead{}
\lfoot{ENT 432 (Fall 2015) - Penn State}
\cfoot{}
\rfoot{\thepage}
\renewcommand{\footrulewidth}{0.4pt}
\title{Group Curation of Collections}
\author{Andrew R. Deans}
\begin{document}
\cleanlookdateon %removed ordinal date
\maketitle
\thispagestyle{fancy}
\section*{Introduction}
You have each spent several weeks collecting, preparing, labeling, identifying, and databasing arthropod specimens. These collections have been graded individually and now, in a team effort, need to be aggregated into a final class collection. This exercise is intended to \textbf{(a)} help you refine your curation skills, \textbf{(b)} bolster your ability to work as a group on a targeted task, and \textbf{(c)} facilitate the incorporation of your specimens into the Frost Entomological Museum collection, where they will become available for research. Below you will find a set of expectations for this unified collection and details about your grade for this assignment.

\section*{Expectations}
The final composite collection should meet the following criteria:
\begin{enumerate}
\item Each unit tray should have no more than one family in it
\item Each unit tray should have a header card that lists the order and family
\item Within each unit tray, the specimens should be sorted to morphospecies
\item A single, unified, parsable (\textit{i.e.}, it passes our databasing test) spreadsheet should accompany the unified collection
\end{enumerate}

\section*{Recommendation}
The team (\textit{i.e.}, your class) can partition the work however it wants, but be sure to spend time with multiple taxa and task types!

\section*{Grading rubric}
You will work as a team to deconstruct individual collections and then reorganize specimens into the unified collection. Each individual, however, should provide a summary of his/her contributions to the process. This report is worth \textbf{50 points} and could take several forms. A table (in a spreadsheet or a document) or narrative with bulleted list are likely the easiest approaches. See the collection guide document for examples of curatorial activities. The 2 points per action guideline is simply a ballpark estimate. Some curation tasks require more effort than others and would be weighted accordingly.
  


\end{document}
