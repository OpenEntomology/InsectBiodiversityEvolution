\documentclass[letterpaper, 11pt]{article}
\usepackage{comment} % enables the use of multi-line comments (\ifx \fi) 
\usepackage{fullpage} % changes the margin
\usepackage{fancyhdr} % for footer
\usepackage{graphicx}
\usepackage[usenames,dvipsnames]{color}
\usepackage[UKenglish]{isodate}% http://ctan.org/pkg/isodate for date format
\def\labelitemi{--}

\definecolor{darkblue}{rgb}{0,0,.6}
\definecolor{darkred}{rgb}{.7,0,0}
\definecolor{darkgreen}{rgb}{0,.6,0}
\definecolor{red}{rgb}{.98,0,0}
\usepackage[colorlinks,pagebackref,pdfusetitle,urlcolor=darkblue,citecolor=darkblue,linkcolor=darkred,bookmarksnumbered,plainpages=false]{hyperref}\pagestyle{fancy}
\renewcommand{\headrulewidth}{0pt}

\lhead{}
\chead{}
\rhead{}
\lfoot{ENT 432 (Fall 2016) - Penn State}
\cfoot{}
\rfoot{\thepage}
\renewcommand{\footrulewidth}{0.4pt}
\title{Insect Collection Requirements}
\author{Andrew R. Deans and Istv\'an Mik\'o}
\begin{document}
\cleanlookdateon %removed ordinal date
\maketitle
\thispagestyle{fancy}
\section*{Basic guidelines}
The specimens in your collection must be acquired during the semester and for this class. Specimens collected during other periods of your life and/or for other purposes, \textit{e.g}., for other classes or from lab cultures, are \textbf{not eligible for credit}. You may trade specimens with your fellow students, but they must be unidentified; you cannot offer specimens you determined already (\textit{e.g}., ``I'll give you a pentatomid for your lucanid''). Immature insects are are eligible for credit but will only count towards 10\% of your grade. Also, there are \textit{no} exceptions to these two policies: 

\begin{enumerate}
\item Your student collection becomes property of the Frost Entomological Museum upon completion of the course. If you collect specimens you want to keep you must use your own prep supplies and storage. These specimens also cannot be turned in as part of your collection.
\item All specimens must be obtained legally. Collecting in national, state, and most local parks requires a permit (\textit{NB}: If you join the Entomological Society of Pennsylvania you can be listed on their state parks permit). Collecting on private land (if you have the owner’s permission) and lands owned by PSU generally does not require a permit.% add http://agsci.psu.edu/hunting
\end{enumerate}

\section*{Equipment}
Students can check out collecting gear, by coordinating with the TA and completing the appropriate paperwork. All equipment must be returned by the last laboratory period in order to complete the course and receive a grade. A separate handout describes equipment issues.

\section*{Preparation of specimens}
Specimens and labels prepared for the collection must conform to the instructions provided in lab; check with your instructors regarding label format variations and alternative preservation or pinning techniques. Slightly damaged specimens may be submitted, especially if the damage is indicative of a biological event (\textit{e.g}., attacked by birds) or research (\textit{e.g}., leg pulled for DNA extraction).

All specimens must be associated with a unique identifier (provided) and databased in a spreadsheet. Here's an example spreadsheet - \url{http://bit.ly/LBYuwf}

A word about the collection data sheet: There are 7 fields that must be complete for every specimen:

\begin{itemize}
\item \textbf{identifier} - This is the barcode number.
\item \textbf{verbatim\_label} - the entire contents of your collecting event label(s) 
\item \textbf{latitude} - in digital degrees, with no degree symbol (``-'' indicates South, West)
\item \textbf{longitude} - in digital degrees, with no degree symbol (``-'' indicates West)
\item \textbf{otu\_name} - the lowest taxonomic name you determined for each specimen
\item \textbf{determiner} - your name
\item \textbf{data\_entry\_by} - your name
\end{itemize}

And one field that is for any notes you think are important (which vial a specimen is in, for example): \textbf{notes}

\section*{Accuracy of data}
Data labels, like all other scientific data, must be accurate (locality, date, collector, \textit{etc}.) An erroneous label, if quoted in print, may have serious consequences that will forever reflect on the scientific integrity of the collector. Careless labeling will be harshly penalized, and data fabrication will result in a \textit{ZERO} for the class. 

\section*{Curation guidelines}
Collections must be neat, clearly labeled, and well-organized. Specimens are typically arranged phylogenetically by order and then alphabetically by family within each order. The layout should be clarified with labels. See \textit{Preparing, labeling, and shipping Hexapoda} document for details.

%%%% from here down will likely change!
\section*{Collection timeline and grading rubric}
\paragraph{Second week of October} Preliminary collections are due. At this time you must have at least 50 specimens prepared, labeled, databased, and appropriately curated (\textit{i.e}., sorted to the best of your taxonomic knowledge and organized as if you were donating them to a museum). See preparation guide for more details. Spreadsheet databases (example: \url{http://bit.ly/LBYuwf}) should be emailed to the instructors on or by this date. This collection is worth up to 50 points, and you will receive a written evaluation that describes areas for improvement.

\paragraph{Second week of November} Final collections are due. Spreadsheet databases (example: \url{http://bit.ly/LBYuwf}) should be emailed to your instructors by this date. This collection is worth up to 150 points, calculated as +3 for each order, +1 for each family, and +0.5 for each extra species within a family (up to 5), assuming the specimens satisfy these criteria:

\begin{enumerate}
\item Appropriately preserved---not too high on the pin, not too low, not too crooked, no pinned immatures, \textit{etc}. See \textit{Preparing, labeling, and shipping Hexapoda} (\url{http://bit.ly/InsectPrep)} for more details
\item Perfectly labeled---The three labels (barcode, locality, determination) are at the appropriate heights, are appropriately sized and cut (not too much white space, right angle cuts). See specimen preparation guide for more details
\item Determined minimally to family
\item Appropriately curated---see Curation Guidelines section above
\item Each accurately databased, with respect to identifier, verbatim\_label, latitude, longitude, otu\_name, determiner, data\_entry\_by
\end{enumerate}

Specimens that match the above criteria but which are misidentified will not count towards the grade. Therefore you should include more specimens than you think you need for 150 points. Poorly labeled or prepared specimens or problems that carry over from the mid-semester check (\textit{i.e}., problems you were alerted to but failed to rectify) will be penalized at the instructor’s discretion.

\paragraph{Second week of December} Integrated class collection due. A unified spreadsheet database should be emailed to your instructors by this date, and each student should email a summary (could be a list or spreadsheet; see below) of his/her contributions (approximately +2 for each curatorial action, up to 50 points) to the instructors. Examples of activities that count towards your curation grade:
\begin{itemize}
\item Correcting a misidentification (and explaining why)
\item Aggregating all specimens of a certain taxon (\textit{e.g}., pulling all the Tachinidae from each individual collection and organizing them into a unit tray)
\item Determining specimens to genus or below (and citing your source of information)
\item Taking a family and sorting it to morphospecies (and briefly describing the characters you used)
\item Making a correction in the database
\end{itemize}


\end{document}
