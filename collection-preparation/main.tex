\documentclass[letterpaper, 11pt]{article}
\usepackage{comment} % enables the use of multi-line comments (\ifx \fi) 
\usepackage{fullpage} % changes the margin
\usepackage{fancyhdr} % for footer
\usepackage[UKenglish]{isodate}% http://ctan.org/pkg/isodate for date format
\usepackage{wrapfig}
\usepackage[font={small,sf}]{caption}
\usepackage{float}%force tables/figs into certain placement
\usepackage{graphicx}%for figures
\usepackage[]{caption}
\usepackage[leftcaption]{sidecap}%for figure captions
\usepackage{subcaption}%for figures
\usepackage{hyperref}%for hyperlinks
\usepackage[font=small,labelfont=bf]{caption}%for captions
\usepackage{natbib}	%for bibliography
\usepackage{placeins}%prevent images from floating into inappropriate sections
\usepackage{tabulary}%for table at the end, to get soft wrapping
%\usepackage[capbesideposition=inside,facing=yes,capbesidesep=quad]{floatrow}


\captionsetup[figure]{labelformat=empty}% redefines the caption setup of the figures environment in the beamer class.

\def\labelitemi{--}

\pagestyle{fancy}
\renewcommand{\headrulewidth}{0pt}

\lhead{}
\chead{}
\rhead{}
\lfoot{ENT 432 (Fall 2016) - Penn State}
\cfoot{}
\rfoot{\thepage}
\renewcommand{\footrulewidth}{0.4pt}
\title{Collection guidelines: How to prepare, label, and database insect specimens}
\author{Open Entomology Project}

\begin{document}
\cleanlookdateon %removed ordinal date
\maketitle
\thispagestyle{fancy}

The insect collection stands as a core exercise for this course, providing future students and researchers with evidence of your hard work and acquired knowledge and with specimens that serve as vouchers for natural history observations. In this light, proper collecting, specimen preparation, and curation methods are fundamental. This handout provides guidance on current best practices. You may also wish to consult the U.S. Department of Agriculture manual, updated by \cite{USDAmanual}, and other resources for more details regarding the approaches described or alluded to below. 

\section*{Supplies and Equipment}
Some of the equipment needed to observe, collect, and preserve insects will be provided by the instructors. Many of these supplies are also available through vendors like Bioquip (\url{https://www.bioquip.com/}):

\begin{itemize}
\item Vials, 4 dram with polyseal caps (Bioquip \# 8804P)
\item 80\% ethanol, for preserving insects
\item Shellac (Bioquip \# 1160), for gluing insects to points
\item Points for point mounting (we have point punches and Bristol board for students to use in lab)
\item Stainless steel, size 2 pins (Bioquip \# 1208S2)
\item label paper, 3$\times$5 cards and cellophane envelopes for Odonata
\item Schmitt box(es)
\item PSUC identifier labels
\item Stereo- and compound microscopes (cannot leave the classroom)
\item Soft-tipped forceps (Bioquip \#'s 4748 or 4750)
\item Waterproof pen for labels; a good option from Bioquip is the size 005 Pigma pen (black) \# 1154E. You can probably find these or their equivalent at an art supply store or even the PSU bookstore.
\item ethyl acetate for kill jar
\end{itemize}

Slide-mounting and double-mounting supplies are available upon request, as is acetone for Odonata specimens. These kinds of specimen preparations will be done in the lab.\\

We have collecting equipment that students members of the Friends of the Frost Entomological Museum can check out for short periods of time. See your instructors about how to check-out this gear. This gear will also be available to everyone during field trips. Sharing is encouraged with this equipment:

\begin{itemize}
\item UV/Hg-vapor lights with white sheet
\item Yellow bowls
\item Malaise trap
\item Aquatic D net and white pans
\item Winkler extractors, litter sifters
\item Berlese funnel
\item Sweep net
\item Spreading board
\end{itemize}

We can also show you how to make your very own:
 
\begin{itemize}
\item aspirator
\item kill jar
\end{itemize}


\section*{Preparing Hexapoda}
A rough guide to preservation preferences---which taxa are pinned vs. slide-mounted, etc.---is provided in the Appendix. See the end of this document.\\

Pinning.---Pinning is the best way to preserve hard-bodied, medium to large hexapods. One should use specially made insect pins, rather than common pins used in sewing and other crafts. Insect pins range in size from 000 (VERY thin and mostly unmanageable) to 7 (very thick and longer than most pins). We recommend sizes 1, 2 (especially), and 3 for use in this class. The best way to pin an insect is to:
 
\begin{enumerate}
\item hold the dead insect between your index finger and thumb
\item pass the pin vertically through the mesonotum, slightly to the right of center (except in Lepidoptera, which gets the pin through the center of the mesonotum), such that it emerges near the right mid coxa
\item slide the pin far enough through the body such that approximately 1 cm of pin is left between the insect and the pin head – this forms the “handle” for manipulating specimens. Pin placement often varies slightly by taxon (see above).
\end{enumerate}

Lepidopterans, especially the larger species, need to have their wings spread by using a spreading board (image source: Goliathus) (see also this demonstration: \url{http://utahbugclub.org/layout/pdf/thebigshow.pdf})...\\

Pointing.---Smaller insects are usually pointed on a piece of acid-free cardstock or Bristol board. Place a drop of alcohol-soluble glue, such as Gelva or shellac, on the tip of a point that’s already been pinned. Touch the point tip to the mesosternum of the insect, usually between the fore and mid coxae. The pointed insect should be oriented in a similar position to that of a pinned insect...\\

Double mounts.---Microlepidopterans need to be double mounted using minuten pins. Essentially this is the same as pinning, except one uses a very small pin (the minuten) to mount the lepidopteran. That mount is then affixed to a normal-sized insect pin via a small piece of foam (see above, right). A more detailed description is provided by Chris Grinter.\\

Alcohol.---Hexapods preserved in \textgreater95\% alcohol are best for DNA extraction, especially if they are kept cold. Unfortunately, high concentrations of alcohol also tend to make specimens a bit fragile, by dehydrating them. Concentrations below 70\% are generally not recommended, as specimens have a better chance of rotting. We use 80\% ethanol in the lab, which works well for nearly any kind of hexapod. Soft-bodied hexapods must always be preserved in alcohol, rather than being pinned or pointed. Adult Lepidoptera should never be preserved in ethanol, as the scales will detach.

Slide Mounting.---See separate slide mounting handout.

\section*{Locality Labels}

Labels can take many forms, but you generally want to adhere to the following formats:

\begin{enumerate}
\item Labels always begin with the country (sometimes in bold or ALL CAPS) and continue with finer scale details of the locality: USA: PA: Centre Co.: etc. One should always include the lat/long, which can now be estimated in programs like Google Earth. Using decimal based degrees is preferred over minutes and/or seconds, as they are easier to database.
\item The date should be formatted such that the month is in lowercase Roman numerals (e.g., October would be ‘x’): 12.x.2010 or 11–12.x.2010 or 11.x–12.xi.2010
\item The collector’s name(s) should also be included, as should the collecting method. Common abbreviations include: MT=Malaise trap, YPT=yellow pan trap, FIT=flight intercept trap, SS=screen sweep. Collectors’ names are sometimes followed by “leg.”, which is short for the Latin ‘lego’, to gather or collect.
\item A sans-serif font (like Arial Narrow) makes the label more readable when the size gets small. Most people use 4 pt for the font size. The finished label should be informative, with a minimal amount of abbreviations, but also reasonably small in size. The information should also be presented in a symmetrical label (i.e., minimize whitespace).
\item Labels that seem to be excessively large (label B) can be cut into two labels. Specimens are prone to multiple labeling from future studies (voucher label, determination label, accession numbers, barcodes, etc.), so it’s desirable to keep the label number to a minimum.
\item Use cotton rag, acid free cardstock for printing labels.
\end{enumerate}

\section{Fluid-preserved Specimens}
Same suggestions apply to labels for fluid-preserved specimens, but one can and should make his/her labels slightly larger (5 or 6 pt) and more elongate. Small labels act like blades and chop up wet-preserved specimens, which are usually relatively soft-bodied.

\section{Determination Labels}
Determination (``det'') labels can also vary in their appearance, but it’s important that they include the taxon name(s), the name of the determiner, and the year (or date) the determination was made. Again, det labels for fluid preserved specimens (label H) should be slightly larger and more elongate:


\section{Preparing Specimens for Shipment}

Adapted from a summary by L. L. Deitz, NC State University.%add link to beetles in the Bush
 

\subsection{Schmitt boxes}

\begin{enumerate}
\item select strong, tight fitting box with suitable pinning substrate – not soft Styrofoam
\item remove fumigant, slides, vials
\item place cardboard on top of Schmitt box (under the lid)
\item make sure pins are well distributed in empty areas of box and are fixed well in substrate
\item brace genitalia vials, large insects, and rotating insects
\item use some type of filler- foam, cotton, newspaper – on top of the cardboard
\item place Schmitt box(es) within a larger package
\item fasten address labels on outside Schmitt box as well as on outside of package
\item ship only 1 box in a larger package to prevent 2 boxes from hitting each other or ship 2 boxes together only if they are securely tied together
\item package with 2 inches on each side between Schmitt box and bigger package, filled with packing material
\end{enumerate}

\subsection*{Slides}

\begin{enumerate}
\item make sure they are double boxed similar to Schmitt box
\item pack slides with coverslips facing each other, form a “brick”
\item place cardboard between opposing coverslips
\item if in slide box, place packing material between top of slides and lid to prevent breakage
\end{enumerate}

\subsection*{Vials}
\begin{enumerate}
\item avoid air bubbles and “burp” vials
\item wrap each vial in paper towel or bubble wrap to separate from other vials
\item shipping ethanol is now heavily regulated and often illegal; consult with safety expert(s) and use UPS (possibly) or FedEx
\end{enumerate}

%cheat sheet for specimen preps
\clearpage
\section*{Appendix. Insect preservation guide}

\begin{figure}[h!]
    \centering
    \begin{subfigure}[b]{0.28\textwidth}
        \includegraphics[width=\textwidth]{Protura}
        \caption{Protura. Andy Murray (CC BY-SA 2.0) \url{https://flic.kr/p/eaKZ2B}}
        \label{fig:protura}
    \end{subfigure}
    ~ 
    \begin{subfigure}[b]{0.32\textwidth}
        \includegraphics[width=\textwidth]{Diplura}
        \caption{Diplura. Marshal Hedin (CC BY-SA 2.0) \url{https://flic.kr/p/e9kGSw}}
        \label{fig:diplura}
    \end{subfigure}
    ~ 
    \begin{subfigure}[b]{0.28\textwidth}
        \includegraphics[width=\textwidth]{Collembola}
        \caption{Collembola. Andy Murray (CC BY-SA 2.0) \url{https://flic.kr/p/bJCR9e}}
        \label{fig:collembola}
    \end{subfigure}
    \caption{\textbf{Entognatha}. Traditionally these hexapods are slide-mounted. For this class, they can be preserved in vials, with ethanol or glycerol as the preservative.}
    \label{fig:entognatha}
\end{figure}

\begin{figure}[h!]
    \centering
    \begin{subfigure}[t]{0.42\textwidth}
        \includegraphics[width=\textwidth]{Archeognatha}
        \caption{Archaeognatha. Henry Lydecker (CC BY-NC 2.0) \url{https://flic.kr/p/dgRtaW}}
        \label{fig:archeognatha}
    \end{subfigure}
    ~ 
    \begin{subfigure}[t]{0.4\textwidth}
        \includegraphics[width=\textwidth]{Zygentoma}
        \caption{Zygentoma. Jean-Rapha\"{e}l Guillaumin (CC BY-SA 2.0) \url{https://flic.kr/p/czibHd}}
        \label{fig:zygentoma}
    \end{subfigure}
    \caption{\textbf{Non-pterygote Insecta}. These insects can be preserved in vials, with ethanol or glycerol as the preservative.}
    \label{fig:winglesshexapods}
\end{figure}

\begin{SCfigure}[][h!]
  \caption{\textbf{Ephemeroptera} (mayflies). Adults and juvenile stages of these insects  are preserved in vials with ethanol or glycerol. Adults are usually quite fragile, so be careful!\\ Photo: Magnus Hagdorn (CC BY-SA 2.0) \url{https://flic.kr/p/ffsx8c}}
  \includegraphics[width=0.45\textwidth]{Ephemeroptera}
\end{SCfigure}

\begin{SCfigure}[][h!]
  \caption{\textbf{Odonata} (dragonflies and damselflies). Odonata should be kept alive in a glassine envelope or paper triangle, until they can be euthanized and subsequently soaked with acetone. Dried specimens are then preserved with a 3$\times$5 card, inside a cellophane envelope. Larvae are always preserved in vials with ethanol or glycerol.\\ Photo: Andy Deans (CC BY 2.0) \url{https://flic.kr/p/oa95N7}}
  \includegraphics[width=0.45\textwidth]{Odonata}
\end{SCfigure}

\begin{SCfigure}[][h!]
  \caption{\textbf{Phasmatodea} (walking stick, stick insects, leaf insects). These insects should be pinned in a way that minimizes their size (legs and antennae close to body). Never spread their wings. Nymphs are always preserved in vials with ethanol or glycerol. \\ Photo: Norman Walsh (CC BY-NC 2.0) \url{https://flic.kr/p/3gcGh5}}
  \includegraphics[width=0.45\textwidth]{Phasmatodea}
\end{SCfigure}

\begin{SCfigure}[][h!]
  \caption{\textbf{Orthoptera} (grasshoppers, crickets, katydids). These insects should be pinned in a way that minimizes their size (legs and antennae close to body). The gut of large specimens should be removed prior to pinning (sharp forceps under the posterior edge of pronotum). Left wings could be spread. Nymphs, crickets (Gryllidae), and other soft-bodied orthopterans go in ethanol.\\ Photo: Andreas Kay (CC BY-NC-SA 2.0) \url{https://flic.kr/p/mR5QLa}}
  \includegraphics[width=0.45\textwidth]{Orthoptera}
\end{SCfigure}

\begin{SCfigure}[][h!]
  \caption{\textbf{Grylloblattodea}. These insects can be preserved in vials, with ethanol or glycerol as the preservative.\\ Photo: Alex Wild (CC0) \url{https://goo.gl/DAU1HJ} (Wikimedia Commons)}
  \includegraphics[width=0.45\textwidth]{Grylloblattodea}
\end{SCfigure}

\clearpage

\begin{SCfigure}[][h!]
  \caption{\textbf{Mantophasmatodea}. These insects can be preserved in vials, with ethanol or glycerol as the preservative.\\ Photo: P. E. Bragg (CC BY-SA 3.0) \url{https://goo.gl/kpnn99} (Wikimedia Commons)}
  \includegraphics[width=0.45\textwidth]{Mantophasmatodea}
\end{SCfigure}

\begin{SCfigure}[][h!]
  \caption{\textbf{Dermaptera} (earwigs). Adult dermapterans should be pinned or pointed. Nymphs are always preserved in vials with ethanol or glycerol.\\ Photo: Mick E. Talbot (CC BY-NC-SA 2.0) \url{https://flic.kr/p/baT4zp}}
  \includegraphics[width=0.45\textwidth]{Dermaptera}
\end{SCfigure}

\begin{SCfigure}[][h!]
  \caption{\textbf{Non-isopteran Dictyoptera} (cockroaches, mantids). These insects should be pinned/pointed in a way that minimizes their size (legs and antennae close to body). Never spread their wings. Genitalia are important for determination, so make sure they’re exposed. Nymphs are always preserved in vials with ethanol or glycerol.}
  \includegraphics[width=0.45\textwidth]{Blattodea}
\end{SCfigure}

\begin{SCfigure}[][h!]
  \caption{\textbf{Dictyoptera: Isoptera} (termites). Termites are always preserved in vials with ethanol or glycerol. \\ Photo: Stevenw12339 (CC BY-NC 2.0) \url{https://flic.kr/p/fAdvLc}}
  \includegraphics[width=0.45\textwidth]{BlattodeaIsoptera}
\end{SCfigure}

\clearpage

\begin{SCfigure}[][h!]
  \caption{\textbf{Embioptera} (webspinners). These insects are preserved in vials with ethanol or glycerol.\\ Photo: Bill \& Mark Bell (CC BY-NC-SA 2.0) \url{https://flic.kr/p/cCJB7S}}
  \includegraphics[width=0.45\textwidth]{Embioptera}
\end{SCfigure}

\begin{SCfigure}[][h!]
  \caption{\textbf{Zoraptera}. These insects are preserved in vials with ethanol or glycerol. \\ Photo: David Maddison (CC BY 3.0) \url{http://goo.gl/hSP3EW} (tolweb.org)}
  \includegraphics[width=0.45\textwidth]{Zoraptera}
\end{SCfigure}

\begin{SCfigure}[][h!]
  \caption{\textbf{Plecoptera} (stoneflies). These insects are preserved in vials with ethanol or glycerol.\\ Photo: Bernard DuPont (CC BY-SA 2.0) \url{https://flic.kr/p/dKDmYT}}
  \includegraphics[width=0.45\textwidth]{Plecoptera}
\end{SCfigure}

\begin{SCfigure}[][h!]
  \caption{\textbf{Thysanoptera} (thrips). Traditionally these insects are slide-mounted. For this class, they can be preserved in vials with ethanol or glycerol.\\ Photo: Katja Schulz (CC BY 2.0) \url{https://flic.kr/p/q6LxBS}}
  \includegraphics[width=0.45\textwidth]{Thysanoptera}
\end{SCfigure}

\clearpage

\begin{SCfigure}[][h!]
  \caption{\textbf{Psocodea} (bark lice, book lice, parasitic lice). Traditionally these insects are slide-mounted. For this class, they can be preserved in vials with ethanol or glycerol.\\ Photo: Ken Schneider (CC BY-NC 2.0) \url{https://flic.kr/p/mX6h9D}}
  \includegraphics[width=0.45\textwidth]{PsocodeaBark}
\end{SCfigure}

\begin{SCfigure}[][h!]
  \caption{\textbf{Hemiptera: Heteroptera, Auchenorrhyncha} (true bugs, hoppers). These insects should be pinned/pointed. Nymphs are always preserved in vials with ethanol or glycerol.\\ Photo: NY State IPM Program (CC BY 2.0) \url{https://flic.kr/p/guTgFA}}
  \includegraphics[width=0.45\textwidth]{HemipteraHeteroptera}
\end{SCfigure}

\begin{SCfigure}[][h!]
  \caption{\textbf{Hemiptera: Sternorrhyncha} (scale insects, aphids). These insects are traditionally slide-mounted. For this class, they can be preserved in vials with ethanol or glycerol.\\ Photo: Jon Sullivan (CC BY-NC 2.0) \url{https://flic.kr/p/oSAp1Y}}
  \includegraphics[width=0.45\textwidth]{HemipteraSternorrhyncha}
\end{SCfigure}

\clearpage

\begin{SCfigure}[][h!]
  \caption{\textbf{Mecoptera} (scorpionflies). These insects should be pinned/pointed. Never spread their wings. Exception: Boreidae (snow scorpionflies) and all larvae should be preserved in vials with ethanol or glycerol.\\ Photo: Orest Shvadchak (CC BY-SA 2.0) \url{https://flic.kr/p/945m9F}}
  \includegraphics[width=0.45\textwidth]{Mecoptera}
\end{SCfigure}

\begin{SCfigure}[][h!]
  \caption{\textbf{Siphonaptera} (fleas). Traditionally these insects are slide-mounted. For this class, they can be preserved in vials with ethanol or glycerol.\\ Photo: AFPMB (CC0) \url{https://flic.kr/p/9bKUYn}}
  \includegraphics[width=0.45\textwidth]{Siphonaptera}
\end{SCfigure}

\begin{SCfigure}[][h!]
  \caption{\textbf{Diptera} (flies, gnats, mosquitoes). These insects should be pinned/pointed in a way that one can view all sclerites (\textit{i.e.}, legs pulled away from body). Wing veins and bristle patterns are important for diagnosis. Exceptions: Small, dainty flies, with soft cuticle (\textit{e.g.}, Cecidomyiidae (gall midges)) and larvae should be preserved in vials with ethanol or glycerol. Also, scaly flies, like mosquitoes, should be double-mounted, with the minuten entering the thorax dextrally.\\ Photo: Troup Dresser (CC BY-NC 2.0) \url{https://flic.kr/p/cYaAPo}}
  \includegraphics[width=0.45\textwidth]{Diptera}
\end{SCfigure}

\clearpage

\begin{SCfigure}[][h!]
  \caption{\textbf{Lepidoptera} (moths, butterflies). These insects should be pinned or double-mounted, depending on their size. Wings must be spread, as color patterns and wing venation are important. Larvae should be preserved in vials with ethanol or glycerol.\\ Photo: Andy Reago \& Chrissy McClarren (CC BY 2.0) \url{https://flic.kr/p/ofEZW6}}
  \includegraphics[width=0.45\textwidth]{Lepidoptera}
\end{SCfigure}

\begin{SCfigure}[][h!]
  \caption{\textbf{Trichoptera} (caddisflies). These insects are preserved in vials with ethanol or glycerol.\\ Photo: Macroscopic Solutions (CC BY-NC 2.0) \url{https://flic.kr/p/o4e7U7}}
  \includegraphics[width=0.45\textwidth]{Trichoptera}
\end{SCfigure}

\begin{SCfigure}[][h!]
  \caption{\textbf{Neuroptera} (lacewings, antlions, mantisflies). These insects should be pinned/pointed in a way that minimizes their size (legs and antennae close to body). Never spread their wings. Larvae are preserved in vials with ethanol or glycerol.\\ Photo: Mick E. Talbot (CC BY 2.0) \url{https://flic.kr/p/6mWUDw}}
  \includegraphics[width=0.45\textwidth]{Neuroptera}
\end{SCfigure}

\begin{SCfigure}[][h!]
  \caption{\textbf{Megaloptera} (dobsonflies, fishflies, alderflies). These insects should be pinned/pointed in a way that minimizes their size (legs and antennae close to body). Never spread their wings. Larvae are preserved in vials with ethanol or glycerol.\\ Photo: Ronald Orosz (CC BY-NC 2.0) \url{https://flic.kr/p/4TEtzq}}
  \includegraphics[width=0.45\textwidth]{Megaloptera}
\end{SCfigure}

\clearpage

\begin{SCfigure}[][h!]
  \caption{\textbf{Raphidioptera} (snakeflies). These insects should be pinned/pointed in a way that minimizes their size (legs and antennae close to body). Never spread their wings. Larvae are preserved in vials with ethanol or glycerol.\\ Photo: Tab Tannery (CC BY-NC-SA 2.0) \url{https://flic.kr/p/92UMz6}}
  \includegraphics[width=0.45\textwidth]{Raphidioptera}
\end{SCfigure}

\begin{SCfigure}[][h!]
  \caption{\textbf{Strepsiptera} (twisted-wing parasites). Traditionally these hexapods are slide-mounted. For this class, they can be preserved in vials with ethanol or glycerol. An infected host (\textit{e.g.}, a vespid wasp) should be preserved according to that taxon’s suggested method.\\ Image: J. Sowerby (CC0), via Biodiversity Heritage Library \url{https://flic.kr/p/a9W5Qa}}
  \includegraphics[width=0.45\textwidth]{Strepsiptera}
\end{SCfigure}

\begin{SCfigure}[][h!]
  \caption{\textbf{Coleoptera} (beetles). These insects should be pinned/pointed. Ventral characteristics of the thorax are important for diagnosis, as are features of the legs, so make sure they are observable. Never spread their wings. Larvae are preserved in vials with ethanol or glycerol.\\ Photo: Gilles San Martin (CC BY-SA 2.0) \url{https://flic.kr/p/hCtNmf}}
  \includegraphics[width=0.45\textwidth]{Coleoptera}
\end{SCfigure}

\begin{SCfigure}[][h!]
  \caption{\textbf{Hymenoptera} (sawflies, wasps, ants, bees). These insects should be pinned/pointed. Exceptions: Small, dainty hymenopterans with soft cuticle (e.g., most Chalcidoidea) and larvae should be preserved in vials with ethanol or glycerol.\\ Photo: Patrick\_K59 (CC BY 2.0) \url{https://flic.kr/p/peXyQ8}}
  \includegraphics[width=0.45\textwidth]{Hymenoptera}
\end{SCfigure}

% adding bibliography here
\bibliographystyle{apalike}
\bibliography{bib}
\end{document}
