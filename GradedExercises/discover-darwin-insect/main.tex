\documentclass[letterpaper, 11pt]{article}
\usepackage{comment} % enables the use of multi-line comments (\ifx \fi) 
\usepackage{fullpage} % changes the margin
\usepackage{fancyhdr} % for footer
\usepackage[UKenglish]{isodate}% http://ctan.org/pkg/isodate for date format
\usepackage[]{natbib}
\usepackage{hyperref}
\usepackage{graphicx}
\usepackage{wrapfig}
\usepackage[normalem]{ulem}%for strikethrough
\usepackage[font={small,sf}]{caption}
\usepackage{epigrafica}%changes default font to epigrafica

\pagestyle{fancy}
\renewcommand{\headrulewidth}{0pt}

\lhead{}
\chead{}
\rhead{}
\lfoot{ENT 432 (Fall 2017) - Penn State}
\cfoot{}
\rfoot{\thepage}
\renewcommand{\footrulewidth}{0.4pt}
\title{Discover your inner Darwin: insect observation and collection}
\author{Open Entomology Project}
\def\labelitemi{--}

\begin{document}
\cleanlookdateon %removed ordinal date
\maketitle
\thispagestyle{fancy}

\section*{Background}
\begin{wrapfigure}{r}{0.4\textwidth}
  \vspace{-24pt}
  \begin{center}
    \includegraphics[width=0.36\textwidth]{Rafinesque}%Dyer https://flic.kr/p/FUBois
  \end{center}
  \vspace{-17pt}
  \caption{A page from Constantine Rafinesque's (1818) field notebook. What would a page from yours look like? See original image: \url{https://flic.kr/p/dbEePV}}
  \vspace{-30pt}
\end{wrapfigure}

Entomology is informed, enriched, and inspired by a broad knowledge of natural history, what \cite{Tewksbury01042014} refer to as the ``fundamental properties of organisms''. We cover natural history extensively in lecture discussions and lab exercises---what insects eat, how they reproduce, how to diagnose taxa, how different groups are related, adaptations they have, how they interact with symbionts, parasites, predators, \textit{etc}. With respect to really understanding and appreciating insects, though, there is no substitute for time spent carefully observing these organisms in their natural environment and then in the lab, under the microscope. \\

\noindent{}For this exercise you will spend time---at least three hours each, in two or more locations---carefully observing and documenting the insect life in a 4 m\textsuperscript{2} area. You will keep extensive field notes about what you see, and you will collect specimens that serve as vouchers for your observations and for the fauna represented in your area. This collection will be extended through subsequent collecting, the details of which are described below.

\section*{Your assignment}

\subsection*{Preparing for Discovery}
Spend 10--15 minutes observing an ant nest, a flower, a spider web, or some other natural object. With your pencil and field notebook, start recording what you see. What kinds of data are relevant to your observations? What objects would you sketch? \\

\noindent{}Now imagine that this exercise was extended into a deep, three-hour (or longer) session, covering 4 m\textsuperscript{2}. Which habitat would you want to observe? What kinds of supplies and equipment would you need? Note that specialized collecting gear is available through the museum. Other questions to consider in your initial proposal:
\begin{itemize}
\item What qualities would you want your study sites to have? How would you incorporate a temporal component? 
\item What are the minimum data you intend to record? How would a typical page in your notebook be organized?
\item How would you collect specimens and label them in a way that they could be associated with the observations you made in your journal? Be sure to read the Collection section below.
\item How would you account for the arthropod fauna you weren't able to observe directly? It's important that you sample your plot extensively, growing your collection beyond those few arthropods you spent time observing. 
\item Make a list of gear you need to bring.
\item Photos, recordings, and video are not required for these observations. If you do have digital content from these sessions, though, how will you make them available for a broad audience and in a sustainable way?
\end{itemize}

\noindent{}Write your ideas in the field notebook, and discuss the approach with your fellow students. This original proposal should be turned in to your instructors. Don't worry if it's rough around the edges! We will follow up on this exercise with a reading and discussion of some observations \cite{Darwin1859} made on ants.

\subsection*{Field Notebook (100 pts.)}
The notebook serves as the record of your natural history observations. It likely will contain substantial data about insects, but it may also include sketches, thoughts, stories, hypotheses, notes about what to do in future observations, \textit{etc}. This component is a substantial portion of your grade for this exercise, and people will have free access to the final product. Any mistakes you make while writing must \textit{NOT} be erased. A single line \sout{threw} through the error is sufficient to indicate a mistake. \textbf{Note on style:} We allow for a more or less free flow of ideas, but each page must follow some basic formatting: your \textit{name} and the \textit{date} in top corner of page, and the \textit{locality} in top middle of each page.\\

\noindent{}We will provide you with a field notebook that is archival and water resistant (Rite In The Rain No: 371FX, 4 5/8 $\times$ 7$''$, 48 pages). \textit{This notebook is property of the Frost Entomological Museum and must be deposited there at the end of the semester.} You should use only pencil or an archival ink (\textit{e.g.}, Sakura\textregistered{ }Pigma Micron\textregistered; \url{http://www.pigmamicron.com/museums}); neither of these is provided. The field notebook should be turned in with your collection on the last day of class.

\subsection*{Collection (150 pts.)}
You should collect specimens for each kind of insect you observed in your plot, especially if they are the subject of content in your field notebook (\textit{i.e.}, they are vouchers for observations). You should also collect specimens of insects you \textit{didn't} see but know occurred in your plot. To complete your collection find specimens of 40 insect families that did \textit{not} occur in your plot, including 10 families we don't cover in lab.\\

\noindent{}Specimens must be determined minimally to family, organized alphabetically by taxon, and accompanied by a spreadsheet \cite{GBIFspreadsheet} (email to instructors). See \cite{FrostSOP03} for more details regarding specimen preparation and labeling. The spreadsheet should be emailed to your instructors and the collection turned in on or before the last day of class. Final drafts should be submitted electronically on or before the last day of class.

\subsection*{Synthesis and highlights (100 pts.)}
During your observation session(s) there was undoubtedly some insect that you found extraordinarily compelling. It exhibited unusual behavior, for example, or had some bizarre phenotype you want to know more about. You should take the relevant specimens, determine them to species, and then research the phenomenon that you found interesting. The results should be summarized in a narrative written for a broad audience---for example, something you would see on a blog---and include references, including the diagnostic tools you used, sketches, photos, \textit{etc}. Ask lots of questions and discuss your ideas with your instructors. Solicit feedback on your narrative as it develops! A draft is due at the first class meeting in November.

\section*{Further reading}
Numerous authors have highlighted the importance of natural history knowledge for the life sciences. \cite{agrawal2014} and \cite{wilcoveeisner2000} provide relatively simple yet compelling examples of the importance of insect observation. See also \cite{Schmidly449} and \cite{Barrows13042016} for discussions of natural history as part of the broader life sciences curriculum. For a celebration of field notes, including examples from ecologists, ethologists, systematists, and other scientists see \cite{canfield2011field}. Be sure also to peruse \cite{roberts2013} for a fascinating read on the effects of deep, patient observation.

\clearpage

\section*{Epilogue}
This handout is part of an open curriculum. Original files are available free for anyone to download, copy, modify, and improve at the Open Entomology GitHub repository \citep{ENT432}, which also provides a mechanism for reporting problems and other feedback:\\
\url{https://github.com/OpenEntomology/InsectBiodiversityEvolution/issues}

% adding bibliography here
\bibliographystyle{apalike}
\bibliography{bib}
\end{document}
