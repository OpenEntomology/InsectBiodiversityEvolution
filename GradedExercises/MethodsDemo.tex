\documentclass[letterpaper, 11pt]{article}
\usepackage{comment} % enables the use of multi-line comments (\ifx \fi) 
\usepackage[letterpaper,margin=0.75in]{geometry}% makes it letter
\usepackage{fullpage} % changes the margin
\usepackage{fancyhdr} % for footer
\usepackage[UKenglish]{isodate}% http://ctan.org/pkg/isodate for date format

\pagestyle{fancy}
\renewcommand{\headrulewidth}{0pt}

\lhead{}
\chead{}
\rhead{}
\lfoot{ENT 432 (Fall 2016) - Penn State}
\cfoot{}
\rfoot{\thepage}
\renewcommand{\footrulewidth}{0.4pt}
\title{Collecting methods demonstration}
\author{Andrew R. Deans}
\begin{document}
\cleanlookdateon %removed ordinal date
\maketitle
\thispagestyle{fancy}
\section*{Introduction}
Specimen collection is fundamental to this class and to the broader scientific enterprise. In this lab, you and your partner(s) will teach your fellow classmates how to use the collecting equipment assigned to you. In return, you will learn from them about a broad array of approaches used to collect arthropods from myriad habitats.

\section*{Methods covered}

\begin{enumerate}
\item beat sheet / sweep net / aspirator / photoeclector
\item collecting aquatic insects 
\item Malaise trap / flight intercept trap
\item pitfall traps, pan traps
\item litter sifting / Winkler extractor / Berlese (Tullgren) funnel
\item light traps, baits
\end{enumerate}

%%%%%%%%% needs grading rubric %%%%%%%

Circle the number you’re responsible for and find the other person(s) you’ll be working with. The Internet offers myriad references for this demonstration, especially archived listservs and various entomology forums. Gibb \& Oseto (2006) also offer a reasonable account of different collecting methods.

\section*{Demonstration approach}
You and your partner(s) are responsible for producing the following educational materials:

\begin{enumerate}
\item A 1-page handout or a Wikipedia page\footnote{A Wikipedia page might exist already for your assigned method(s), but if you can substantially improve it that is a valid option to replace the handout. Please talk to your instructors about this option if you’re interested.} for each assigned method, including photos, illustrations, text, and references. Please submit your materials to the instructors 24 hours before the demonstration.
\item A 10-minute (maximum) demonstration for your classmates that includes: how to use the method, how the method works, logistical considerations. Alternatively, you can prepare a video (for YouTube, Vimeo, or some other outlet) that we will watch in class. Do not, however, use other people's videos as a substitute for your own presentation.
\end{enumerate}

\begin{thebibliography}{9}
\bibitem{ref} Gibb, T. \& C. Oseto (2006) \textit{Arthropod Collection and Identification: Laboratory and Field Techniques}. Academic Press. 336 pp. ISBN-13: 978-0123695451

\end{thebibliography}

\end{document}
