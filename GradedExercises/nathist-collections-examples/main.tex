\documentclass[letterpaper, 11pt]{article}
\usepackage{comment} % enables the use of multi-line comments (\ifx \fi) 
\usepackage{fullpage} % changes the margin
\usepackage{fancyhdr} % for footer
\usepackage[UKenglish]{isodate}% http://ctan.org/pkg/isodate for date format

\pagestyle{fancy}
\renewcommand{\headrulewidth}{0pt}

\lhead{}
\chead{}
\rhead{}
\lfoot{ENT 432 (Fall 2016) - Penn State}
\cfoot{}
\rfoot{\thepage}
\renewcommand{\footrulewidth}{0.4pt}
\title{Natural history collections: digitization and use}
\author{Open Entomology Project}

\begin{document}
\cleanlookdateon %removed ordinal date
\maketitle
\thispagestyle{fancy}
\section*{Background}
Imagine that you've been recruited to contribute to a white paper about the importance and utility of natural history collections. A major granting agency is considering a substantial investment---tens of millions of dollars---in this type of collection, with the intention of making the specimens and their associated data more accessible for science (broadly defined). Your responsibility is to provide an entomological perspective: how have arthropod specimens and data been used in science (and maybe even beyond science)? The richer your collective set of examples is, the more likely this agency is to commit funding to collections. 

\section*{Your assignment}
Below you'll find a reference \cite{suarez2004} that should ``prime the pump''---\textit{i.e.}, get you started on a search for articles that support the notion that collections are invaluable resources. Can you \textbf{(1)} find an article that describes research that would impossible without natural history collections? Can you then \textbf{(2)} find one or two articles that yield \textit{novel}\footnote{That is, uses of specimens that surprised you} uses of natural history specimens?\\

In the contexts of your selected articles and the collections you're building for this course, think about what ``accessibility'' means. This major granting agency might allocate tens of millions of dollars to ``digitization''. What are the data that should be made available and how? Is this a good investment?\vspace{2.5cm}



\begin{thebibliography}{9}
\bibitem{suarez2004} Suarez, Andrew V. and Neil D. Tsutsui (2004) The value of museum collections for research and society. \textit{BioScience} 54(1): 66--74. DOI: 10.1641/0006-3568(2004)054[0066:TVOMCF]2.0.C
\end{thebibliography}

\end{document}
\begin{enumerate} 
\item{Chelicerata} 
\item{Xiphosura}  
\item{Arachnida}  
\item{Acari}
\item{Araneae}  
\item{Opiliones}  
\item{Pseudoscorpiones} 
\item{Myriapoda}
\item{Chilopoda}
\item{Diplopoda}
\end{enumerate}
\section*{Bigger picture}
We covered more taxa than those in the list above. Was there one that stood out to you as especially compelling? Can you spell it correctly and describe its relevance to our understanding of evolution, behavior, and morphology?\\

\noindent{}What was the first terrestrial arthropod? How do we know it was terrestrial?\\ %add something about papers
