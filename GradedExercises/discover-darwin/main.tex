\documentclass[letterpaper, 11pt]{article}
\usepackage{comment} % enables the use of multi-line comments (\ifx \fi) 
\usepackage{fullpage} % changes the margin
\usepackage{fancyhdr} % for footer
\usepackage[UKenglish]{isodate}% http://ctan.org/pkg/isodate for date format
\usepackage[]{natbib}
\usepackage{hyperref}
\usepackage{graphicx}
\usepackage{wrapfig}
\usepackage[font={small,sf}]{caption}

\pagestyle{fancy}
\renewcommand{\headrulewidth}{0pt}

\lhead{}
\chead{}
\rhead{}
\lfoot{ENT 432 (Fall 2016) - Penn State}
\cfoot{}
\rfoot{\thepage}
\renewcommand{\footrulewidth}{0.4pt}
\title{Discover your inner Darwin}% need sexier title
\author{Open Entomology Project}

\begin{document}
\cleanlookdateon %removed ordinal date
\maketitle
\thispagestyle{fancy}

\section*{Background}
\begin{wrapfigure}{r}{0.5\textwidth}
  \vspace{-20pt}
  \begin{center}
    \includegraphics[width=0.38\textwidth]{Rafinesque}
  \end{center}
  \vspace{-16pt}
  \caption{A page from Constantine Rafinesque's (1818) field notebook. What would a page from yours look like? See original image: \url{https://flic.kr/p/dbEePV}}
  \vspace{-38pt}
\end{wrapfigure}
Entomology is informed, enriched, and inspired by a broad knowledge of natural history, what \cite{Tewksbury01042014} refer to as the ``fundamental properties of organisms''. We cover insect natural history extensively in lecture discussions and lab exercises---what insects eat, how they reproduce, how to diagnose taxa, how taxa are related, adaptations they have, how they interact with symbionts, parasites, predators, \textit{etc}. With respect to really understanding and appreciating insects, though, there is no substitute for time spent carefully observing these organisms in their natural environment. 

For this exercise you will spend time---at least 3--4 hours---carefully observing and documenting the insect life in a 9 m\textsuperscript{2} area. You will keep extensive field notes about what you see, and you will collect specimens that serve as vouchers for your observations.

\section*{Your assignment}

\subsection*{Initial Proposal (20 pts.)}
Spend five or ten minutes observing an ant nest or spider web. If you had a pencil and paper what kinds of data would you be recording? What would you sketch? How would you collect specimens and then label them in a way that they could be associated with the observations you made in your journal? Now imagine that this 5--10 minute exercise was extended into a deep, 4-hour session, covering a much larger area. What kinds of supplies and equipment would you want to have with you? Keep in mind that this 9 m\textsuperscript{2} plot will be chosen by your instructors and could be in almost any habitat (a meadow, at the edge of a pond, leaf litter and fallen logs in an eastern deciduous forest, \textit{etc}.) Your instructors will provide the field notebook and basic locality data for the site. Collecting gear is also available.\\

\noindent{}Write your ideas out on a piece of paper and discuss them with your partner. This original proposal should be turned in to your instructors.

\subsection*{Refined Proposal (30 pts.)}
After discussing your approach with your partner did you make any changes? Summarize your conversation (one or two paragraphs) and append that to a refined proposal to be turned in to your instructors.

\subsection*{Field Notebook (100 pts.)}
The notebook serves as the record of your natural history observations. It likely will contain substantial data about insects, but it may also include sketches, thoughts, stories, hypotheses, notes about what to do in future observations, \textit{etc}. This component is a substantial portion of your grade for this exercise, and people will have free access to the final product. Any mistakes you make while writing \textit{must not be erased}. A single line through the error is sufficient to indicate a mistake.\\

\noindent{}We will provide you with a field notebook that is archival and water resistant (Rite In The Rain No: 371FX, 4 5/8 $\times$ 7'', 48 pages). This notebook is property of the Frost Entomological Museum and must be deposited there at the end of the semester. You should use only pencil or an archival ink (\textit{e.g.}, Sakura\textregistered{ }Pigma Micron\textregistered; \url{http://www.pigmamicron.com/museums}); neither of these are provided. 

\subsection*{Field Collection (50 pts.)}
You should collect specimens for each kind of insect you observed, especially if they are the subject of content in your field notebook (\textit{i.e.}, they are vouchers for observations). These specimens must be determined minimally to family, and the collection should be accompanied by a spreadsheet. See the collection guidelines handouts for more details regarding specimen preparation and collection organization.

\subsection*{Highlights (50 pts.)}
During each natural history session there were undoubtedly one or two insects that you found extraordinarily compelling. They exhibited unusual behavior, for example, or had some unusual phenotype you want to know more about. You should take these specimens and determine them to species and then research the phenomenon that you found interesting. The results should be summarized in a narrative written for a broad audience---for example, something you would see on a blog---and include references, sketches, photos, \textit{etc}.

% adding bibliography here
\bibliographystyle{apalike}
\bibliography{bib}
\end{document}

Given this paper - http://dx.doi.org/10.1093/biosci/biw043 - perhaps we need a natural history exercise.

(0) Student thinks about his/her background and interests. Based on these and a brief experiment (e.g., watch a spider web for 10 mins) s/he develops a proposed workflow or approach to observing and documenting insect natural history. What materials and supplies will s/he need? How will specimens be matched to observations in a notebook? What kinds of data will be collected? A proposal (1 pg) is written and given to a partner.

(1) Students vet each other's proposal and provide constructive feedback. A refined proposal is turned in for a grade.

(2) Instructors set up 4 m2 areas in different habitats. Students are told which ones they are to spend 4 hours observing/collecting/documenting the insects in it.

(3) Students are then told which one to move to for new observations.

(4) After 2 or 3 iterations student medidates on these observations, discusses them with partner, then chooses 2 or 3 (or 1?) observations s/he wants to develop knowledge on.

(5) The specimens from those key observations get determined to species, there is some literature search, and development of a presentation that is given eventually to the group.

(6) To complement this initial collection, which likely has low diversity relative to all possible insects that occur in PA, students dip into bulk samples taken from similar habitats (e.g., a Malaise trap set up in same meadow for a week). They each need to pull out insects that represent 20 families they didn't observe during their exercise.

(7) The entire collection and raw field notes are turned in, and the stories are presented to the class
