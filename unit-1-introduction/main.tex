\documentclass[11pt,letterpaper]{article}
\usepackage{comment} % enables the use of multi-line comments (\ifx \fi) 
%
\usepackage[letterpaper,margin=0.75in]{geometry}
\usepackage{fullpage} % changes the margin
\usepackage{fancyhdr} % for footer
\usepackage[UKenglish]{isodate}% http://ctan.org/pkg/isodate for date format

\pagestyle{fancy}
\renewcommand{\headrulewidth}{0pt}

\lhead{}
\chead{}
\rhead{}
\lfoot{ENT 432 (Fall 2016) - Penn State}
\cfoot{}
\rfoot{\thepage}
\renewcommand{\footrulewidth}{0.4pt}
\title{Module 1 - What are arthropods? What is systematics?}
\author{Andrew R. Deans and Istv\'an Mik\'o}
\begin{document}
\cleanlookdateon %removed ordinal date
\maketitle
\thispagestyle{fancy}
\section*{Introduction}
In this module we will discuss systematics as a field of research, touch on the importance of specimens \cite{voucher} and introduce ourselves to the taxonomic focus of this course: Arthropoda. Read the course syllabus \cite{syllabus} and familiarize yourself with the grading policy, dates of exams and due dates for assignments, \textit{etc}.
\section*{Concepts}
Could you explain the following terms to one of your peers?

\begin{enumerate} 
\item{nomenclature} 
\item{classification}  
\item{phylogenetics}  
\item{natural history}
\item{Arthropoda}  
\item{Hexapoda}  
\item{Insecta}  
\end{enumerate}

\section*{Bigger picture}
Can you draw a phylogeny and interpret its meaning?

\begin{thebibliography}{9}%do we want to add references to these handouts?
\bibitem{syllabus} Deans, A. R. \textit{Insect Biodiversity and Evolution}. Fall 2016 syllabus.
\bibitem{voucher} Turney S., Cameron E. R., Cloutier C. A., Buddle C. M. (2015) Non-repeatable science: assessing the frequency of voucher specimen deposition reveals that most arthropod research cannot be verified. \textit{PeerJ} 3:e1168 DOI: 10.7717/peerj.1168
\end{thebibliography}

\end{document}

