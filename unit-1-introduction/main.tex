\documentclass[11pt,letterpaper]{article}
\usepackage{comment} % enables the use of multi-line comments (\ifx \fi) 
%
\usepackage[letterpaper,margin=0.75in]{geometry}
\usepackage{fullpage} % changes the margin
\usepackage{fancyhdr} % for footer
\usepackage[UKenglish]{isodate}% http://ctan.org/pkg/isodate for date format
\usepackage{epigrafica}%changes default font to epigrafica
\usepackage{hyperref}%for URLs
\usepackage[]{natbib}
\newcommand*{\doi}[1]{\href{http://dx.doi.org/#1}{doi: #1}}% links DOI

\pagestyle{fancy}
\renewcommand{\headrulewidth}{0pt}

\lhead{}
\chead{}
\rhead{}
\lfoot{ENT 432 (Fall 2016) - Penn State}
\cfoot{}
\rfoot{\thepage}
\renewcommand{\footrulewidth}{0.4pt}
\title{Unit 1 - What are arthropods? What is systematics?}
\author{Open Entomology Project}

\begin{document}
\cleanlookdateon %removed ordinal date
\maketitle
\thispagestyle{fancy}
\section*{Introduction}
In this module we will discuss systematics as a field of research, and we'll touch on the importance of specimens as data, references, and vouchers \citep{vouchers}. Read the course syllabus and familiarize yourself with the grading policy, dates of exams and due dates for assignments, \textit{etc}. We'll also discuss some ethical issues regarding the collection of organisms, and we'll get started on the collection exercise---Discover your inner Darwin.

\section*{Test yourself}
In each unit handout you will find questions, typically prefaced with the unit number (``4-1'' and ``4-2'', for the first two questions in unit 4, for example). You should answer these in your lab notebook, which will be checked weekly. Your lab notebook grade is based on your effort in answering these questions, as well as how neat and organized the notes are, rather than whether your answer was correct.\\

\noindent\textbf{Question 1-1:} Could you explain the following terms to one of your peers?
\begin{enumerate} 
\item Arthropoda
\item classification
\item Hexapoda
\item Insecta
\item nomenclature
\item phylogenetics 
\end{enumerate}

\noindent{}\textbf{Question 1-2:} Can you draw a phylogeny and interpret its meaning? Why are these diagrams important? \cite{baum2008phylogenics} provide a helpful guide.\\

\noindent{}\textbf{Question 1-3:} Could you explain to a friend how this class is graded?\\

\noindent{}\textbf{Question 1-4:} Can you describe three ethical issues we need to consider when collecting organisms? If you were to edit the Insect Collector's Oath \citep{oath} how would you change it?\\

\section*{Epilogue}
This handout is part of an open curriculum. Original files are available free for anyone to download, copy, modify, and improve at the Open Entomology GitHub repository \citep{ENT432}, which also provides a mechanism for reporting problems and other feedback:\\
\url{https://github.com/OpenEntomology/InsectBiodiversityEvolution/issues}

% adding bibliography here
\bibliographystyle{myplainnat}
\bibliography{bib}

\end{document}
