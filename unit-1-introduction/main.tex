\documentclass[11pt,letterpaper]{article}
\usepackage{comment} % enables the use of multi-line comments (\ifx \fi) 
%
\usepackage[letterpaper,margin=0.75in]{geometry}
\usepackage{fullpage} % changes the margin
\usepackage{fancyhdr} % for footer
\usepackage[UKenglish]{isodate}% http://ctan.org/pkg/isodate for date format
\usepackage{epigrafica}%changes default font to epigrafica
\usepackage{hyperref}%for URLs
\usepackage[]{natbib}
\newcommand*{\doi}[1]{\href{http://dx.doi.org/#1}{doi: #1}}% links DOI

\pagestyle{fancy}
\renewcommand{\headrulewidth}{0pt}

\lhead{}
\chead{}
\rhead{}
\lfoot{ENT 432 (Fall 2016) - Penn State}
\cfoot{}
\rfoot{\thepage}
\renewcommand{\footrulewidth}{0.4pt}
\title{Unit 1 - What are arthropods? What is systematics?}
\author{Open Entomology Project}

\begin{document}
\cleanlookdateon %removed ordinal date
\maketitle
\thispagestyle{fancy}
\section*{Introduction}
In this module we will discuss systematics as a field of research, and we'll touch on the importance of specimens as data, references, and vouchers \citep{vouchers}. Read the course syllabus and familiarize yourself with the grading policy, dates of exams and due dates for assignments, \textit{etc}. We'll also discuss some ethical issues regarding the collection of organisms, and we'll learn how to make specimen preps.

\section*{Concepts}
Could you explain the following terms to one of your peers?
\begin{enumerate} 
\item Arthropoda
\item classification
\item Hexapoda
\item Insecta
\item nomenclature
\item phylogenetics 
\end{enumerate}



\section*{Bigger picture}
Can you draw a phylogeny and interpret its meaning? Why are these diagrams important?\\

\noindent{}Could you explain to a friend how this class is graded?\\

\noindent{}Can you describe a few ethical issues we need to consider when collecting organisms? If you were to edit the Insect Collector's Oath \citep{oath} how would you change it?

% adding bibliography here
\bibliographystyle{myplainnat}
\bibliography{bib}

\end{document}
