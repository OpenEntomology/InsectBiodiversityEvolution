\documentclass[11pt]{article}
%\usepackage{geometry}
\usepackage[letterpaper,inner=1.5cm,outer=1.5cm,top=2.5cm,bottom=2.5cm]{geometry}
\pagestyle{empty}
\usepackage{graphicx}
\usepackage{fancyhdr, lastpage, bbding, pmboxdraw}
\usepackage[usenames,dvipsnames]{color}
\usepackage{textcomp}
\usepackage{gensymb}
\usepackage[table,xcdraw]{xcolor}
\usepackage{natbib}
\newcommand*{\doi}[1]{\href{http://dx.doi.org/#1}{doi: #1}}% links DOI

\usepackage{array}% for centering cell in longtable
\usepackage{caption}%for table title
\usepackage{longtable}%for tables that span multiple tables
\definecolor{darkblue}{rgb}{0,0,.6}
\definecolor{darkred}{rgb}{.7,0,0}
\definecolor{darkgreen}{rgb}{0,.6,0}
\definecolor{red}{rgb}{.98,0,0}
\usepackage[colorlinks,pagebackref,pdfusetitle,urlcolor=darkblue,citecolor=darkblue,linkcolor=darkred,bookmarksnumbered,plainpages=false]{hyperref}
\renewcommand{\thefootnote}{\fnsymbol{footnote}}

\pagestyle{fancy}
\renewcommand{\headrulewidth}{0pt}
\lhead{}
\chead{}
\rhead{}
\lfoot{ENT 432 (Fall 2015) - Penn State}
\cfoot{}
\rfoot{\thepage}
\renewcommand{\footrulewidth}{0.4pt}

%%%%%%%%%%%% LISTING %%%
\usepackage{listings}
\usepackage{caption}
\DeclareCaptionFont{white}{\color{white}}
\DeclareCaptionFormat{listing}{\colorbox{gray}{\parbox{\textwidth}{#1#2#3}}}
%\captionsetup[lstlisting]{format=listing,labelfont=white,textfont=white}
\usepackage{verbatim} % used to display code
\usepackage{fancyvrb}
\usepackage{acronym}
\usepackage{amsthm}
\VerbatimFootnotes % Required, otherwise verbatim does not work in footnotes!

\definecolor{OliveGreen}{cmyk}{0.64,0,0.95,0.40}
\definecolor{CadetBlue}{cmyk}{0.62,0.57,0.23,0}
\definecolor{lightlightgray}{gray}{0.93}

\lstset{
%language=bash,                          % Code langugage
basicstyle=\ttfamily,                   % Code font, Examples: \footnotesize, \ttfamily
keywordstyle=\color{OliveGreen},        % Keywords font ('*' = uppercase)
commentstyle=\color{gray},              % Comments font
numbers=left,                           % Line nums position
numberstyle=\tiny,                      % Line-numbers fonts
stepnumber=1,                           % Step between two line-numbers
numbersep=5pt,                          % How far are line-numbers from code
backgroundcolor=\color{lightlightgray}, % Choose background color
frame=none,                             % A frame around the code
tabsize=2,                              % Default tab size
captionpos=t,                           % Caption-position = bottom
breaklines=true,                        % Automatic line breaking?
breakatwhitespace=false,                % Automatic breaks only at whitespace?
showspaces=false,                       % Dont make spaces visible
showtabs=false,                         % Dont make tabls visible
columns=flexible,                       % Column format
morekeywords={__global__, __device__},  % CUDA specific keywords
}

%%%%%%%%%%%%%%%%%%%%%%%%%%%%%%%%%%%%
\begin{document}
\begin{center}
{\Large \textsc{Insect Biodiversity and Evolution (ENT 432)}}
\end{center}
\begin{center}
Department of Entomology, Pennsylvania State University\end{center}

\begin{center}
\rule{6in}{0.4pt}
\begin{minipage}[t]{.74\textwidth}
\begin{tabular}{llcl}
\textbf{Instructor:} & Dr. Andrew R. Deans & &  \href{mailto:adeans@psu.edu}{adeans@psu.edu} \\
\textbf{Co-instructor:} & Dr. Istv\'an Mik\'o & &  \href{mailto:istvan.miko@gmail.com}{istvan.miko@gmail.com} \\
%\textbf{Teaching Assistant:} & Asher Jones? & &  \href{mailto:carolyntrietsch@gmail.com}{carolyntrietsch@gmail.com} \\
\textbf{Teaching Assistant:} & TBD & &  \href{mailto:TBD@psu.edu}{TBD@psu.edu} \\
\textbf{Lecture:} & T Th 1:00--2:15 PM & & 320 Wagner Building \\
\textbf{Lab:} & T Th 2:30--4:20 PM & & 131 Headhouse III \\
\end{tabular}
\end{minipage}
\rule{6in}{0.4pt}
\end{center}
\vspace{.3cm}
\setlength{\unitlength}{1in}
\renewcommand{\arraystretch}{2}

\paragraph{Office Hours:} By appointment.

\paragraph{Course Website:} \url{http://deanslab.org/ent432/}

\paragraph{Preamble}  This course is designed to teach students about insect taxonomy, evolutionary relationships, collection and preservation techniques, morphology, and natural history. We'll focus mostly on adult forms and emphasize insects found in Pennsylvania. In the lab, students will learn how to handle specimens, use diagnostic keys, and identify insects by sight. Collection techniques will be honed during field trips. 

\paragraph{Learning Outcomes} Upon completion of this course students should be able to:
\begin{enumerate}
\item Label a generalized hexapod diagram with external anatomy terms
\item Name and sight-identify all local hexapod orders and several common local families
\item Solve taxonomic problems and describe how species and other taxa are named and described, \textit{i.e}., understand the fundamentals of taxonomic practice
\item Draw a phylogenetic tree of relationships between hexapod orders
\item Teach others how to read a phylogenetic tree and know what kinds of data are used to estimate trees, how those data are analyzed, and what it means to be monophyletic
\item Describe key innovations and life history strategies of major hexapod lineages
\item Teach others how to collect, preserve, and transport hexapod specimens and describe why this process is important
\item Teach others about the natural history of insects more generally
\end{enumerate} 

\noindent To realize these educational outcomes students are expected to:
\begin{itemize}
\item Attend every lecture and lab
\item Participate in discussions
\item Ask questions
\end{itemize}

\paragraph{Textbook} %\footnotemark
There is no required textbook. Instructors will discuss and make available resources that facilitate specimen diagnosis and learning about the evolution of these organisms.
%\begin{itemize}
%\item Evans, A. V. (2007) \textit{National Wildlife Federation Field Guide to Insects and Spiders \& Related Species of North America} ISBN: 978-1402741531
%\item Gibb, T. \& C. Oseto (2006) \textit{Arthropod Collection and Identification: Laboratory and Field Techniques}. Academic Press. 336 pp. ISBN: 978-0123695451
%\item Grimaldi, D. \& M. S. Engel (2005) \textit{Evolution of the Insects}. Cambridge University Press, New York. 772 pp. ISBN: 978-0521821490.
%\item Johnson, N. F. \& C. A. Triplehorn (2004) \textit{Borror and DeLong's Introduction to the Study of Insects}. Cengage Learning, 7th edition. 888 pp. ISBN: 978-0030968358
%\item Whitfield, J. \& A. H. Purcell III (2012) \textit{Daly and Doyen's Introduction to Insect Biology and Diversity}. Oxford University Press. 736 pp. ISBN: 978-0-19-538067-5
%\end{itemize} 

\begin{center}
\begin{longtable}{>{\centering\arraybackslash}p{1cm}>{\centering\arraybackslash}p{5cm}>{\centering\arraybackslash}p{5cm}>{\centering\arraybackslash}p{3cm}} %>{\centering\arraybackslash}m{1cm}
%\caption*{A rough outline of what topics will be covered when. Note that readings listed in the far right column are required.}\\
\hline
\textbf{Date} & \textbf{Lecture} & \textbf{Lab} & \textbf{Reading(s)} \\
\hline
\endfirsthead
\multicolumn{4}{c}%
{\tablename\ \thetable\ -- \textit{Continued from previous page}} \\
\hline
\textbf{Date} & \textbf{Lecture} & \textbf{Lab} & \textbf{Reading(s)} \\
% * <carolyntrietsch@gmail.com> 2016-01-31T17:09:25.817Z:
%
% > Reading
%
% If you want them to actually do the readings, you'll need to incorporate them into each class lecture as well, maybe have a 10 minute discussion on each paper. Or you could also assign each student to give a synopsis on one of the papers during the semester.
%
% ^.
\hline
\endhead
\hline \multicolumn{4}{r}{\textit{Continued on next page}} \\
\endfoot
\hline
\endlastfoot
8/25 & (1) Expectations, assignments, arthropods, systematics & specimen preparation and handling clinic & \cite{Tewksbury01042014} \\ 
8/27 & \multicolumn{2}{c}{(2) Arthropod morphology (131 Headhouse III)} & \cite{Giribet201596} \\ 
9/1 & (3) History of classification, principles of evolution & (2) Arthropod morphology & \cite{EngelKristensen2013} \\ 
9/3 & Principles of evolution, systematics & Characterizing phenotypic diversity; diagnostic tools & \cite{BaumOffner2008} \\
9/8 & Fossils, early Arthropoda, terrestrialization & Fossils, Onychophora, and non-insect Arthropoda & \cite{TrautweinEtAl2012} \\ 
9/10 & Non-insect arthropods & Non-insect Arthropoda cont'd. & \cite{DunlopEtAl2013,MartinezDelclos200419} \\ 
9/15 & Entognatha, apterygote Insecta & Palaeoptera & \cite{Dunlop2010124, MiyazawaEtAl2014} \\ 
9/17 &  & Polyneoptera & \cite{VanHook2008R547,crustAppend} \\
9/22 &  & Polyneoptera, review & \cite{FavretEtAl2015}  \\ 
9/24 & Origin of wings, Palaeoptera, Neoptera & \textbf{Lab practical 1}, work on collections & \cite{engelwing} \\ 
9/29 & Dermaptera, Zoraptera, Plecoptera, Orthoptera & Thysanoptera, Psocodea, Hemiptera & \cite{straussHearing} \\ 
10/1 & Dictyoptera, Xenonomia, Eukinolabia & Hemiptera & \cite{nalepaTermite,CockroachesDinos} \\
10/6 & Thysanoptera, Psocodea & Hemiptera, Hymenoptera & \cite{ReedEtAl2015} \\ 
10/8 & Hemiptera, review & Hymenoptera & \cite{hemipteraTymbals} \\ 
10/13 & \textbf{Mid-term exam} & Hymenoptera & - \\ 
10/15 & Holometabola, Hymenoptera & Neuropterida, \textbf{preliminary collections due} & \cite{metamorphosis1999} \\
10/20 & Hymenoptera & \textbf{Lab practical 2}, work on collections & \cite{BIJ351} \\ 
10/22 & Hymenoptera, Neuropterida & Coleoptera & \cite{crespi1994} \\ 
10/27 & Coleoptera & Coleoptera & \cite{Smith20150060} \\ 
10/29 & Coleoptera & Coleoptera, Strepsiptera & \cite{Emlen15052007} \\
11/3 & Coleoptera, Strepsiptera & Diptera & \cite{Kathirithamby2015} \\ 
11/5 & Mecoptera, Siphonaptera & Diptera & \cite{Zhu2015129} \\ 
11/10 & Diptera & \textbf{Individual collections due} & \cite{Wiegmann05042011} \\ 
11/12 & Diptera & Diptera & \cite{Deora03022015} \\
\textit{11/17} & \multicolumn{2}{c}{\textit{ESA annual meeting}} & - \\ 
11/19 & Lepidoptera & Lepidoptera & \cite{Janzen29062010} \\ 
\textit{11/24} & \multicolumn{2}{c}{\textit{Thanksgiving}} & - \\
\textit{11/26} & \multicolumn{2}{c}{\textit{Thanksgiving}} & - \\
12/1 & Lepidoptera & Lepidoptera & \cite{Wahlberg2013} \\ 
12/3 & Lepidoptera, Trichoptera & Trichoptera & \cite{SYEN12016} \\
12/8 & Natural history collections, future directions & \textbf{Lab practical 3}, work on class collection & \cite{Kress12122014,kleijn2008} \\ 
12/10 & Review & \textbf{Class collection and spreadsheet due, along with curation summary} & \cite{Misof07112014} \\ 
\end{longtable}
\end{center}

\paragraph{Grading Policy} Letter grades follow the usual scale: A = 93--100\%, A- = 90--92.9\%, B+ = 87--89.9\%, B = 83--86.9\%, B- = 80--82.9\%, C+ = 76--79.9\%, C = 70--75.9\%, D = 60--69.9\%, F \textless 60\%.

\paragraph{Graded Components}
\begin{center} \begin{minipage}{3.8in}
\begin{flushleft}
Collecting methods demonstration   \dotfill 50 pts.  \\
Lecture exam   \dotfill 100 pts.  \\
Lab practicals (n=3)   \dotfill 150 pts.  \\
Insect collection    \dotfill 250 pts.  \\
Final exam   \dotfill 150 pts.  \\
Participation\footnote{Based on attendance (35\%), participation in lecture discussions and quizzes (25\%), and answering questions in lab handouts (40\%).}    \dotfill 100 pts.  \\
\textbf{Total}   \dotfill \textbf{800 pts.} 
\end{flushleft}
\end{minipage}
\end{center}


\paragraph{Academic Honesty} Penn State and the College of Agricultural Sciences take violations of academic integrity very seriously. Faculty, alumni, staff and fellow students expect each student to uphold the University’s standards of academic integrity both in and outside of the classroom.

Academic integrity is the pursuit of scholarly activity in an open, honest and responsible manner. Academic integrity is a basic guiding principle for all academic activity at The Pennsylvania State University, and all members of the University community are expected to act in accordance with this principle. Consistent with this expectation, students should act with personal integrity, respect other students' dignity, rights and property, and help create and maintain an environment in which all can succeed through the fruits of their efforts. Academic integrity includes a commitment not to engage in or tolerate acts of falsification, plagiarism, misrepresentation or deception. Such acts of dishonesty violate the fundamental ethical principles of the University community and compromise the worth of work completed by others. See Faculty Senate Policy 49‐20 (\url{http://www.psu.edu/ufs/policies/47-00.html#49-20}) and G‐9 Procedures (\url{http://www.psu.edu/dept/oue/aappm/G-9.html}) and the PSU Code of Conduct (\url{http://studentaffairs.psu.edu/conduct/codeofconduct}).

Academic Integrity Guidelines for the College of Agricultural Sciences can be found at: \url{http://agsci.psu.edu/students/resources/academic-integrity}

A lack of knowledge or understanding of the University’s Academic Integrity policy and the types of actions it prohibits and/or requires does not excuse one from complying with the policy.

\paragraph{Statement of Nondiscrimination} The Pennsylvania State University is committed to the policy that all persons shall have equal access to programs, facilities, admission, and employment without regard to personal characteristics not related to ability, performance, or qualifications as determined by University policy or by state or federal authorities. The Pennsylvania State University does not discriminate against any person because of age, ancestry, color, disability or handicap, national origin, race, religious creed, sex, sexual orientation, or veteran status.

\paragraph{Disability Access Statement} Penn State welcomes students with disabilities into the University's educational programs. Every Penn State campus has an office for students with disabilities. The Office for Disability Services (ODS) Web site provides contact information for every Penn State campus: \url{http://equity.psu.edu/ods/dcl}. For further information, please visit the Office for Disability Services Web site: \url{http://equity.psu.edu/ods}. 

In order to receive consideration for reasonable accommodations, you must contact the appropriate disability services office at the campus where you are officially enrolled, participate in an intake interview, and provide documentation: \url{http://equity.psu.edu/ods/guidelines}. If the documentation supports your request for reasonable accommodations, your campus disability services office will provide you with an accommodation letter. Please share this letter with your instructors and discuss the accommodations with them as early in your courses as possible. You must follow this process for every semester that you request accommodations.

\bibliographystyle{myplainnat}%%

\bibliography{refs}

\end{document} 
