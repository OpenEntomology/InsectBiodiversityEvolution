\documentclass[11pt]{article}
%\usepackage{geometry}
\usepackage[letterpaper,inner=1.5cm,outer=1.5cm,top=2.5cm,bottom=2.5cm]{geometry}
\pagestyle{empty}
\usepackage{graphicx}
\usepackage{fancyhdr, lastpage, bbding, pmboxdraw}
\usepackage[usenames,dvipsnames]{color}
\usepackage{textcomp}
\usepackage{gensymb}
\usepackage[table,xcdraw]{xcolor}
\usepackage{natbib}
\usepackage{epigrafica}%changes default font to epigrafica

\newcommand*{\doi}[1]{\href{http://dx.doi.org/#1}{doi: #1}}% links DOI

\usepackage{array}% for centering cell in longtable
\usepackage{caption}%for table title
\usepackage{longtable}%for tables that span multiple tables
\definecolor{darkblue}{rgb}{0,0,.6}
\definecolor{darkred}{rgb}{.7,0,0}
\definecolor{darkgreen}{rgb}{0,.6,0}
\definecolor{red}{rgb}{.98,0,0}
\usepackage[colorlinks,pagebackref,pdfusetitle,urlcolor=darkblue,citecolor=darkblue,linkcolor=darkred,bookmarksnumbered,plainpages=false]{hyperref}
\renewcommand{\thefootnote}{\fnsymbol{footnote}}

\pagestyle{fancy}
\renewcommand{\headrulewidth}{0pt}
\lhead{}
\chead{}
\rhead{}
\lfoot{ENT 432 (Fall 2016) - Penn State}
\cfoot{}
\rfoot{\thepage}
\renewcommand{\footrulewidth}{0.4pt}

%%%%%%%%%%%% LISTING %%%
\usepackage{listings}
\usepackage{caption}
\DeclareCaptionFont{white}{\color{white}}
\DeclareCaptionFormat{listing}{\colorbox{gray}{\parbox{\textwidth}{#1#2#3}}}
%\captionsetup[lstlisting]{format=listing,labelfont=white,textfont=white}
\usepackage{verbatim} % used to display code
\usepackage{fancyvrb}
\usepackage{acronym}
%\VerbatimFootnotes % Required, otherwise verbatim does not work in footnotes!

\definecolor{OliveGreen}{cmyk}{0.64,0,0.95,0.40}
\definecolor{CadetBlue}{cmyk}{0.62,0.57,0.23,0}
\definecolor{lightlightgray}{gray}{0.93}
\def\labelitemi{--}


\lstset{
%language=bash,                          % Code langugage
basicstyle=\ttfamily,                   % Code font, Examples: \footnotesize, \ttfamily
keywordstyle=\color{OliveGreen},        % Keywords font ('*' = uppercase)
commentstyle=\color{gray},              % Comments font
numbers=left,                           % Line nums position
numberstyle=\tiny,                      % Line-numbers fonts
stepnumber=1,                           % Step between two line-numbers
numbersep=5pt,                          % How far are line-numbers from code
backgroundcolor=\color{lightlightgray}, % Choose background color
frame=none,                             % A frame around the code
tabsize=2,                              % Default tab size
captionpos=t,                           % Caption-position = bottom
breaklines=true,                        % Automatic line breaking?
breakatwhitespace=false,                % Automatic breaks only at whitespace?
showspaces=false,                       % Dont make spaces visible
showtabs=false,                         % Dont make tabls visible
columns=flexible,                       % Column format
morekeywords={__global__, __device__},  % CUDA specific keywords
}

%%%%%%%%%%%%%%%%%%%%%%%%%%%%%%%%%%%%
\begin{document}
\begin{center}
{\huge \textsc{Insect Biodiversity and Evolution (ENT 432)}}
\end{center}
\begin{center}
Department of Entomology, Pennsylvania State University\end{center}

\begin{center}
\rule{6in}{0.4pt}
\begin{minipage}[t]{.74\textwidth}
\begin{tabular}{llcl}
\textbf{Instructor:} & Dr. Andrew R. Deans & &  \href{mailto:adeans@psu.edu}{adeans@psu.edu} \\
\textbf{Co-instructor:} & Dr. Istv\'an Mik\'o & &  \href{mailto:istvan.miko@gmail.com}{istvan.miko@gmail.com} \\
%\textbf{Teaching Assistant:} & Asher Jones? & &  \href{mailto:carolyntrietsch@gmail.com}{carolyntrietsch@gmail.com} \\
\textbf{Teaching Assistant:} & Asher Jones & &  \href{mailto:abj5150@psu.edu}{abj5150@psu.edu} \\
\textbf{Lecture:} & T Th 12:05--1:20 PM & & 110 ASI Building \\
\textbf{Lab:} & T Th 1:30--3:20 PM & & 131 Headhouse III \\
\end{tabular}
\end{minipage}
\rule{6in}{0.4pt}
\end{center}
\vspace{.3cm}
\setlength{\unitlength}{1in}
\renewcommand{\arraystretch}{2}

This course is designed to teach students about insect taxonomy, evolutionary relationships, collection and preservation techniques, morphology, and natural history. We'll focus mostly on adult forms and emphasize insects found in Pennsylvania. In the lab, students will learn how to handle specimens, use diagnostic keys, and identify insects by sight. Collection techniques will be honed during field trips. 

\section*{Learning Outcomes} 
Upon completion of this course students should be able to:
\begin{itemize}
\item Label a generalized hexapod diagram with external anatomy terms
\item Name and sight-identify all local hexapod orders and several common local families
\item Solve taxonomic problems and describe how species and other taxa are named and described, \textit{i.e}., understand the fundamentals of taxonomic practice
\item Draw a phylogenetic tree of relationships between hexapod orders
\item Teach others how to read a phylogenetic tree and know what kinds of data are used to estimate trees, how those data are analyzed, and what it means to be monophyletic
\item Describe key innovations and life history strategies of major hexapod lineages
\item Teach others how to collect, preserve, and transport hexapod specimens and describe why this process is important
\item Teach others about the natural history of insects more generally
\end{itemize} 

\noindent To realize these educational outcomes students are expected to:
\begin{itemize}
\item Attend every lecture and lab
\item Participate in discussions
\item Ask questions
\end{itemize}

\section*{Logistics} 
\paragraph{Office Hours:} By appointment.

\paragraph{Course Website:} \url{http://deanslab.org/ent432/} Note that source files for teaching materials are available through our GitHub repo - \url{https://github.com/OpenEntomology/InsectBiodiversityEvolution} - which also provides a mechanism for submitting issues: \url{https://github.com/OpenEntomology/InsectBiodiversityEvolution/issues} 

\paragraph{Textbook} 
There is no required textbook. Instructors will discuss and make available resources that facilitate specimen diagnosis and learning about the evolution of these organisms.

\paragraph{Organization}
The course content is partitioned across 15 units, each of which has its own (more or less) structured curriculum, including lecture/discussion components and lab exercises:

\begin{enumerate}
\item What are arthropods? What is systematics?
\item Understanding arthropod phenotypes
\item Systematics and Evolution, past and present
\item Early arthropods, fossils, terrestrialization 
\item Non-hexapod Arthropoda
\item Non-pterygote Hexapoda
\item Paleoptera, Plecoptera, origin of wings
\item Polyneoptera
\item Acercaria
\item Holometabolous development, Hymenoptera
\item Neuropterida
\item Coleoptera, Strepsiptera
\item Antliophora
\item Amphiesmenoptera
\item Natural History Collections
\end{enumerate}

\noindent{}Dispersed across these units are topics (``phenomena'') that transcend taxa---\textit{i.e.}, they will be discussed at multiple points during the semester, as appropriate. Examples include leaf mining and other herbivory strategies, galls, mimicry/aposematism, sound production, weapons, and parasitism.

\paragraph{Grading Policy} Letter grades follow the usual scale: A = 93--100\%, A- = 90--92.9\%, B+ = 87--89.9\%, B = 83--86.9\%, B- = 80--82.9\%, C+ = 76--79.9\%, C = 70--75.9\%, D = 60--69.9\%, F \textless 60\%.

\paragraph{Graded Components}
\begin{center} \begin{minipage}{3.8in}
\begin{flushleft}
Discover your inner Darwin   \dotfill 350 pts.  \\
Lab practicals (n=3)   \dotfill 150 pts.  \\
Lab notebook \dotfill 100 pts.\\
Final exam   \dotfill 100 pts.  \\
Participation\footnote{Based on attendance (35\%) and participation in lecture discussions and quizzes (65\%)}    \dotfill 100 pts.  \\
\textbf{Total}   \dotfill \textbf{800 pts.} 
\end{flushleft}
\end{minipage}
\end{center}
$\gg\gg\gg\gg$ Note that fabricated specimen data and natural history observations will result in a \textit{ZERO} (0) for the Discover your inner Darwin exercise. There are \textit{NO EXCEPTIONS} to this rule. $\ll\ll\ll\ll$

\paragraph{Academic Honesty} Penn State and the College of Agricultural Sciences take violations of academic integrity very seriously. Faculty, alumni, staff and fellow students expect each student to uphold the University’s standards of academic integrity both in and outside of the classroom.

Academic integrity is the pursuit of scholarly activity in an open, honest and responsible manner. Academic integrity is a basic guiding principle for all academic activity at The Pennsylvania State University, and all members of the University community are expected to act in accordance with this principle. Consistent with this expectation, students should act with personal integrity, respect other students' dignity, rights and property, and help create and maintain an environment in which all can succeed through the fruits of their efforts. Academic integrity includes a commitment not to engage in or tolerate acts of falsification, plagiarism, misrepresentation or deception. Such acts of dishonesty violate the fundamental ethical principles of the University community and compromise the worth of work completed by others. See Faculty Senate Policy 49‐20 (\url{http://www.psu.edu/ufs/policies/47-00.html#49-20}) and G‐9 Procedures (\url{http://www.psu.edu/dept/oue/aappm/G-9.html}) and the PSU Code of Conduct (\url{http://studentaffairs.psu.edu/conduct/codeofconduct}).

Academic Integrity Guidelines for the College of Agricultural Sciences can be found at: \url{http://agsci.psu.edu/students/resources/academic-integrity}

A lack of knowledge or understanding of the University’s Academic Integrity policy and the types of actions it prohibits and/or requires does not excuse one from complying with the policy.

\paragraph{Statement of Nondiscrimination} The Pennsylvania State University is committed to the policy that all persons shall have equal access to programs, facilities, admission, and employment without regard to personal characteristics not related to ability, performance, or qualifications as determined by University policy or by state or federal authorities. The Pennsylvania State University does not discriminate against any person because of age, ancestry, color, disability or handicap, national origin, race, religious creed, sex, sexual orientation, or veteran status.

\paragraph{Disability Access Statement} Penn State welcomes students with disabilities into the University's educational programs. Every Penn State campus has an office for students with disabilities. The Office for Disability Services (ODS) Web site provides contact information for every Penn State campus: \url{http://equity.psu.edu/ods/dcl}. For further information, please visit the Office for Disability Services Web site: \url{http://equity.psu.edu/ods}. 

In order to receive consideration for reasonable accommodations, you must contact the appropriate disability services office at the campus where you are officially enrolled, participate in an intake interview, and provide documentation: \url{http://equity.psu.edu/ods/guidelines}. If the documentation supports your request for reasonable accommodations, your campus disability services office will provide you with an accommodation letter. Please share this letter with your instructors and discuss the accommodations with them as early in your courses as possible. You must follow this process for every semester that you request accommodations.

%\bibliographystyle{myplainnat}%%

%\bibliography{refs}

\end{document} 