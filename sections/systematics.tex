\section{Introduction}
Systematists are charged with describing, naming, estimating evolutionary history, and classifying organisms. Here we cover the history of classification, so that you can understand the context of systematics today. \cite{grimaldi2005evolution} and \cite{EngelKristensen2013} together provide a thorough review of this history if you're interested in more details.

Here we remind ourselves of certain core concepts in evolution, which will facilitate our understanding of how Arthropoda radiated and evolved. We'll also discuss the kinds of data and analytical approaches (and even philosophies) we use to estimate evolutionary history (\textit{i.e}., phylogeny).

\section{History, concepts, and people to know}
We'll open this unit with a discussion and slideshow about the history of systematics, as it relates to arthropods, and a reintroduction to evolutionary concepts. Take notes and think about the following questions before moving on to lab-oriented exercises.

\subsection{Systematists}
If asked to describe the accomplishments of these people in 1--2 sentences could you do it? Could you arrange these figures chronologically?

\begin{enumerate} 
\item{Aristotle} 
\item{Charles Darwin}  
\item {Johann C. Fabricius}
\item {Willi Hennig}
\item Niels P. Kristensen
\item{Carolus Linn\ae{us}}  
\item{Maria Sibylla Merian}  
\end{enumerate}

\subsection{Concepts}
Think about the concepts and questions below. Could you write a couple sentences that explain each term? Can you provide examples? Cycle through the hands-on exercises and come back to this section later. 

\begin{enumerate} 
\item classification
\item nomenclature
\item phylogenetics 
\item{natural selection} 
\item{adaptation}  
\item{species}  
\item {adaptive radiation}
\item {monophyletic \textit{vs}. polyphyletic}  
\item {homology \textit{vs}. homoplasy}  
\item {ancestral/plesiomorphic \textit{vs}. derived/apomorphic}
\end{enumerate}

\noindent{}Can you draw a phylogeny and interpret its meaning? Why are these diagrams important? \cite{baum2008phylogenics} provide a helpful guide.\vspace{3mm}

\noindent{}Can you describe three technologies or tools that allowed for major leaps forward in insect taxonomy? (You can probably think of more than three.)\vspace{3mm}

\noindent{}We discussed three types of data that are typically used to estimate phylogenies: phenotype (including behavior), ecology, and molecules (DNA, proteins, \textit{etc}.) Can you briefly describe some advantages and limitations of each data type?\vspace{3mm}

\noindent{}Can you describe the differences between a phenetic and a cladistic approach to phylogeny? What is our current approach to character and phylogeny interpretation?\vspace{3mm}

\section{Classify your specimens}
As you learned earlier, the external features of an arthropod, \textit{i.e}., the phenotypes we can easily observe through a microscope, are the primary source of characters we use to diagnose arthropods and to hypothesize evolutionary relationships. The goal of this lab is to review the morphological concepts you learned in the last unit and then to develop \latinword{characters} (descriptions of phenotypic variation) that help you understand and communicate about Arthropoda. Take some time to sort your specimens into groups.\vspace{3mm}

\begin{theo} 
{}How many groups did you make, and what body parts provided the evidence or characters to justify your groups?\end{theo}

\subsection{Anatomy refresher}
(10--15 min) Look at your lab notebook, and review your answers to questions in the last chapter. Let's see if we can write formal definitions for some of the concepts we learned. Here is the typical structure of a \textit{genus differentia}: A \textit{species} is a \textit{genus} that \textless{}some description of how it differs from other species in that genus\textgreater{}. Here is an example: A microscope is a piece of equipment that allows one to see small things. ``Microscope'' is the species here, and it belongs in the broader class (the \textit{genus}) ``piece of equipment''. Microscope differs from other pieces of equipment by being the one that allows one to see small things (\textit{i.e}., this is the \textit{differentia}).\vspace{3mm}

\begin{theo} 
{}Can you generate a \textit{genus differentia} definition for this anatomical concept:\vspace{3mm}

A \latinword{head} is a \makebox[30 mm]{\hrulefill} that \hrulefill.\vspace{3mm} 

\noindent{}What about for \latinword{ocellus}, \latinword{antenna}, \latinword{pedicel}, \latinword{mandible}, \latinword{maxilla}, \latinword{leg}, \latinword{fore leg}, \latinword{wing}, \latinword{fore wing}, \latinword{mesothorax}?\vspace{3mm}

\noindent{}Note that this is how Linnaeus structured his species descriptions.
\end{theo}

\subsection{Character and matrix development}
(30 min) Think of a \latinword{character} as a relatively broad class of possible phenotypes for a body part. ``Flagellomere shape'', ``femur color'', and ``claw presence'' are three examples of characters. Character states are more specific versions of the character, and they're typically numbered arbitrarily from 0 to $\infty$:\vspace{3mm}

Flagellomere shape:	0=flagellum round, 1=flagellum square\vspace{3mm}

Femur color: 0=femur red, 1=femur orange, 2=femur black\vspace{3mm}

Claw presence: 0=present, 1=absent\vspace{3mm}

With this list of characters, and a list of the taxa you're examining, we can score a matrix:
\begin{table}[H]
\centering
\label{my-label}
\begin{tabular}{lccc}
                             & \multicolumn{1}{l}{flagellomere shape} & \multicolumn{1}{l}{femur color} & \multicolumn{1}{l}{claw presence} \\ \cline{2-4} 
\multicolumn{1}{l|}{insect A} & \multicolumn{1}{c|}{0}                 & \multicolumn{1}{c|}{0}          & \multicolumn{1}{c|}{1}            \\ \cline{2-4} 
\multicolumn{1}{l|}{insect B} & \multicolumn{1}{c|}{1}                 & \multicolumn{1}{c|}{2}          & \multicolumn{1}{c|}{1}            \\ \cline{2-4} 
\multicolumn{1}{l|}{insect C} & \multicolumn{1}{c|}{0}                 & \multicolumn{1}{c|}{1}          & \multicolumn{1}{c|}{0}            \\ \cline{2-4} 
\end{tabular}
\end{table}

\noindent{}We can compute across matrices like this using software that provides us with phylogenetic trees and software that generates diagnostic tools (\textit{e.g}., multi-entry keys).\vspace{3mm}

\begin{theo}[systematics3] 
{}Choose eight of your specimens, and see how many characters you can create for them. Mimic the descriptive style you see in the examples above, and score your taxa in a matrix. How do you account for homology?\end{theo}

\subsection{Dichotomous keys}
(30 min) Dichotomous keys are common diagnostic tools, and you will almost certainly use several in your efforts to determine specimens in your collection. They are composed from character sets, like the table you generated above and typically look like this (adapted from \cite[][page 151]{borror1989introduction}:\vspace{3mm}

\begin{adjustwidth}{1cm}{}
1. With well-developed wings (adults) \dotfill{} 2\\
1$'$. Wingless or with wings vestigial or rudimentary (nymphs, larvae, and some adults) \dotfill{} 30\vspace{3mm}

\noindent{}2(1). Wings membranous, not hardened or leathery \dotfill{} 3\\
2$'$. Front wings hardened or leathery, at least at base; hind wings, if present, usually membranous \dotfill{}24
\end{adjustwidth}\vspace{3mm}

\noindent{}Think of it almost as a series of if-then statements. If your specimen has well-developed wings then go to line 2. If those wings are membranous then go to line 3.\vspace{3mm}

\noindent{}Look through the keys provided on the workbench. See any patterns? Try running a few of your specimens through the key.\vspace{3mm}

\begin{theo} 
{}Write a dichotomous key using the characters you developed above. What kinds of characters would you use first, those that are easy to understand or those that separate the taxa into similarly-sized groups?\end{theo}\vspace{3mm}

\subsection{Thinking phylogenetically}
(30 min) Look at your list of characters and ask yourself how many provide evidence concerning the evolution of these organisms. Which states are ancestral and which ones are derived, and why? How many do you think are purely diagnostic, offering no usable information for understanding evolution?\vspace{3mm}

\begin{theo}[systematics5] 
{}Draw a phylogenetic tree that represents your hypothesis for the evolutionary history of your specimens. Your instructors will demonstrate how to do this on the board. Map your character states on the tree. Do your data support your tree? Can you see evidence of convergent evolution?\end{theo}

\subsection{A few words about taxonomic ranks}
In many cases, one can learn a lot about zoological names just by looking at the endings. Linnean classification is organized hierarchically into \latinword{ranks}, which you might remember from biology class: kingdom, phylum, class, order, family, genus, and species. For the honey bee, the nested classification looks like this:\vspace{3mm}

Animalia $\leftarrow$ Arthropoda $\leftarrow$ Insecta $\leftarrow$ Hymenoptera $\leftarrow$ Apidae $\leftarrow$ \textit{Apis} $\leftarrow$ \textit{mellifera}\vspace{3mm}

\noindent{}Note that genus and species are always in a different font style, for example \textit{italics} or \underline{underlined}. When referencing a species one usually include both genus and species, plus the taxon's author and the year it was published. For the honey bee, the proper reference is \textit{Apis mellifera} Linnaeus, 1758. When the species name has been transferred to another genus since its original description, the author and year are put in parentheses. The black carpenter ant, for example was originally described by De Geer in 1773 as \textit{Formica pensylvanica}. The species was later transferred to \textit{Camponotus} and is now referred to as \textit{Camponotus pennsylvanicus} (De Geer, 1773).\vspace{3mm}

\noindent{}As our understanding and our naming of species has increased, so has the number of ranks: ... order, suborder, infraorder, superfamily, family, subfamily, tribe, subtribe, genus, subgenus, and so on. Some of these ranks (\textit{i.e.}, ``family group'' ranks) have suffixes that indicate their rank. For example, -oidea always refers to superfamily, -idae to family, -inae to subfamily, -ini to tribe, and -ina to subtribe.

\begin{theo}[ranks] 
{}Some taxonomists describe ranks as arbitrary and even burdensome, while others see them as critically useful. Entomologists tend to be in the latter camp, and yet the family-level classification in many orders remains highly unstable (as you will see!). Should we adhere to a rigid system of ranks? \end{theo}

\subsection{Solving nomenclatural problems}
We'll discuss the \cite{iczn} and its Code, portions of which are in the appendix of this handout. Read the following descriptions of nomenclatural problems, and see if you can find solutions in the pieces of the Code given to you in this handout (see Appendix I).\vspace{3mm}

\begin{theo}[systematics6] 
{}(1) Linn\ae{}us described the honey bee in 1758, naming it \textit{Apis mellifera}. Joe Schmoe insists on calling these insects \textit{Honeyus beeus}, however, because he found this name in an even older volume (1725), by another author. Joe thinks \textit{Honeyus beeus} should have priority, as spelled out in Article 23. What is the correct name? Which article resolves this conundrum?\vspace{3mm}

\noindent{}(2) Jane Doe tries to publish a new species name with a modern twist---a crane fly named \textit{Tipula p0k3m0norum} Doe, 2022. The peer review process informs her that this is not possible. Which article do her peers reference in their critique and why?\vspace{3mm}

\noindent{}(3) Frustrated with the high expense and slow rate of scientific publication (and the constant rejection of manuscripts by his peers), Randy Bobandy decides to self-publish all new species. He prints his manuscripts in his mom's basement, avoids peer-review, and mails them to several local libraries to be archived. Are these new names available? Why or why not?
\end{theo}

\section{Specimen data and accesibility}
The next set of exercises introduce you to some basic \latinword{biodiversity informatics}. You've started collecting and sorting insects, undoubtedly from numerous localities. You've also made specimen preparations and labeled each one according to collection best practices. As part of this process you've also established a database of specimens in your collection: a spreadsheet with columns for locality, collector, identifier, \textit{etc}. You should be using the prescribed spreadsheet format for this class, but Appendix II describes how to make non-standard spreadsheets accessible.\vspace{3mm}

\noindent{}We'll talk a bit about biodiversity data standards and tools and other resources that facilitate data sharing. 

\section*{Glossary}
We'll be using terms you may not be familiar with. Some of these concepts are provided below, and we'll go over them together. If you have a new one to add raise your hand!
\begin{enumerate} 
\item {Darwin Core}
\item {GBIF}
\item {metadata}
\item{TDWG} 
\end{enumerate}

\section*{Your data}
Take a look at this label, which probably looks similar to the labels you've created for your specimens:\vspace{3mm}

\begin{labelfontsmall}
\tiny
\noindent{USA: PA: Centre County: \\ Pine Grove Mills, 40.730, \\ -77.884, $\pm$250m 15.iv.2023 \\ A.R. Deans, sifted litter}
\end{labelfontsmall}
\normalsize\vspace{5mm}

\begin{theo}[systematics7] 
{}What kinds of data do you see represented? How many kinds are there? List them.\vspace{3mm}

\noindent{}If you had these data from millions of specimens---all 35+ million insects in the Smithsonian Institution, for example---what kinds of hypotheses could you test? See if you can think of three example research questions.
\end{theo}

\section*{Biodiversity data in research}
Biodiversity science is incredibly rich, with respect to the array of research questions and the data types that can be applied to them. We'll discuss a few examples of research that relies on collections data. As we talk, think about the minimum data required for these kinds of questions.

\section*{Biodiversity data standards}
How can we aggregate data from hundreds (or thousands!) of natural history collections to test your hypotheses? You can imagine that natural history collections across the world might each have a different approach to recording data; for example Museum A might record dates of collection as ``2023-08-31", while Museum B uses ``31-08-2023'' and Museum C records dates as ``31.viii.2023''.\vspace{3mm}

\noindent{}Fortunately there are established biodiversity data standards and tools that facilitate sharing. We'll look at those developed by TDWG, especially the Darwin Core and associated resources. Your homework: Open a Web browser and navigate to \url{https://dwc.tdwg.org/terms/}. This massive list almost definitely includes data types you use, along with dozens more that may or may not be relevant to your collection.\vspace{3mm}

\noindent{}Find the following terms and read their definitions: \latinword{catalogNumber}, \latinword{recordedBy}, \\\latinword{eventDate}, \latinword{samplingProtocol}, \latinword{stateProvince} and \latinword{locality}, \latinword{decimalLatitude}, \\\latinword{decimalLongitude},  \latinword{scientificName}. Do any of these sound familiar?\vspace{3mm}

\section*{More thoughts on biodiversity data}
Now that you have an idea of how to create a set of files that contribute to the greater scientific enterprise it's time to think about how to extend and enrich your data sets. We'll discuss some of these issues as a group.
\begin{itemize}
\item Many journals (\textit{e.g.}, PLoS) and granting agencies want information about specimen provenance and collecting permits. How would you incorporate that into your spreadsheet or archive?
\item All of your specimens will be deposited at the Frost Entomological Museum. How do we specify that in your file?
\item Do any of you have images or videos? How do we associate them with specimen records?
\item How do we associate specimens, for example a parasitoid and its host? 
\item How do we account for a range of dates, as we might see with a Malaise trap?
\item How to explain our approach to georeferencing (\textit{i.e.}, finding a latitude and longitude for each specimen)? Do you know what geodetic datum means?
\item Can biodiversity data be copyrighted?
\end{itemize}

\section*{\textit{ProTip}\texttrademark}
GBIF provides structured spreadsheets as Microsoft Excel files: 
\url{https://ipt.gbif.org/manual/en/ipt/latest/occurrence-data}. These files can be imported through any of the myriad Integrated Publishing Toolkit (IPT) instances.

\clearpage

\section*{Appendix. Portions of the International Code of Zoological Nomenclature}

The following text was copied from the ICZN online edition (\url{https://www.iczn.org/}). It has been simplified for the exercises in this handout. Not all articles are included.

\subsection*{Article 3. Starting point.} 

The date 1 January 1758 is arbitrarily fixed in this Code as the date of the starting point of zoological nomenclature.

\subsection*{Article 5. Principle of Binominal Nomenclature.}

\paragraph*{5.1. Names of species.} The scientific name of a species, and not of a taxon of any other rank, is a combination of two names (a binomen), the first being the generic name and the second being the specific name. The generic name must begin with an upper-case letter and the specific name must begin with a lower-case letter.

\subsection*{Article 8. What constitutes published work.}
A work is to be regarded as published for the purposes of zoological nomenclature if it complies with the requirements of this Article and is not excluded by the provisions of Article 9.

\paragraph*{8.1. Criteria to be met.} A work must satisfy the following criteria:

\begin{itemize}
    \item \textbf{8.1.1.} it must be issued for the purpose of providing a public and permanent scientific record,
    \item \textbf{8.1.2.} it must be obtainable, when first issued, free of charge or by purchase, and
    \item \textbf{8.1.3.} it must have been produced in an edition containing simultaneously obtainable copies by a method that assures
    \begin{itemize}
        \item \textbf{8.1.3.1.} numerous identical and durable copies, or
        \item \textbf{8.1.3.2.} widely accessible electronic copies with fixed content and layout.
    \end{itemize}
\end{itemize}


\subsection*{Article 9. What does not constitute published work.}
Notwithstanding the provisions of Article 8, none of the following constitutes published work within the meaning of the Code:

\paragraph*{9.1.} after 1930 handwriting reproduced in facsimile by any process;

\paragraph*{9.2.} after 1985, works produced by hectographing or mimeographing;

\paragraph*{9.3.} before 1986 and after 2012, works issued on optical discs;

\paragraph*{9.4.} photographs as such;

\paragraph*{9.5.} proof sheets;

\paragraph*{9.6.} microfilms;

\paragraph*{9.7.} acoustic records made by any method;

\paragraph*{9.8.} labels of specimens;

\paragraph*{9.9.} preliminary versions of works accessible electronically in advance of publication (see Article
21.8.3);

\paragraph*{9.10.} materials issued primarily to participants at meetings (e.g. symposia, colloquia, congresses, or workshops), including abstracts and texts of presentations or posters;

\paragraph*{9.11.} text or illustrations distributed by means of electronic signals (e.g. via the Internet), except those fulfilling the requirements of Articles 8.1 and 8.5.

\paragraph*{9.12.} facsimiles or reproductions obtained on demand of an unpublished work [Art. 8], even if previously deposited in a library or other archive.

\subsection*{Article 11. Requirements [for the availability of a name].}
To be available, a name or, where relevant, a nomenclatural act must satisfy the following provisions:

\paragraph*{11.1. Publication.} The name or nomenclatural act must have been published, in the meaning of Article 8, after 1757.

\paragraph*{11.2. Mandatory use of Latin alphabet.} A scientific name must, when first published, have been spelled only in the 26 letters of the Latin alphabet (taken to include the letters j, k, w and y); the presence in a name when first published of diacritic and other marks, apostrophes or ligatures, or a hyphen, or a numeral in a compound species-group name, does not render the name unavailable (for corrections, see Articles 27 and 32.5.2).

\paragraph*{11.3. Derivation.} Providing it meets the requirements of this Chapter, a name may be a word in or derived from Latin, Greek or any other language (even one with no alphabet), or be formed from such a word. It may be an arbitrary combination of letters providing this is formed to be used as a word.

\subsection*{Article 14. Anonymous authorship of names and nomenclatural acts.}
A new name or nomenclatural act published after 1950 with anonymous authorship is not thereby made available.

\subsection*{Article 18. Inappropriate and tautonymous names.}
The availability of a name is not affected by inappropriateness or tautonymy.

\subsection*{Article 23. Principle of Priority.}

\paragraph*{23.1. Statement of the Principle of Priority.} The valid name of a taxon is the oldest available name applied to it, unless that name has been invalidated or another name is given precedence by any provision of the Code or by any ruling of the Commission. For this reason priority applies to the validity of synonyms, to the relative precedence of homonyms, the correctness or otherwise of spellings, and to the validity of nomenclatural acts (such as acts taken under the Principle of the First Reviser and the fixation of name-bearing types).

\subsection*{Article 61. Principle of Typification.}

\paragraph*{61.1. Statement of the Principle of Typification.} Each nominal taxon in the family, genus or species groups has actually or potentially a name-bearing type. The fixation of the name-bearing type of a nominal taxon provides the objective standard of reference for the application of the name it bears.

\subsection*{Article 76. Type locality.}

\paragraph*{76.1. Definition.} The type locality of a nominal species-group taxon is the geographical (and, where relevant, stratigraphical) place of capture, collection or observation of the name-bearing type; if there are syntypes and no lectotype has been designated, the type locality encompasses the localities of all of them.

\subsection*{Article 78. Powers and duties of the Commission.}

\paragraph*{78.1. Plenary Power.} The Commission is empowered, by a resolution of the IX International Congress of Zoology (1913) and ratified in subsequent Codes by its successors, under conditions specified in Article 81 to suspend the application in a particular case of any provision of the Code except those in the present and next succeeding Chapter. The course to be followed is decided by the Commission under this plenary power and its ruling is published in an Opinion [Art. 80.2].
\thispagestyle{empty}