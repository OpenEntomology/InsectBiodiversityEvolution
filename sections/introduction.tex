The diagnosis, classification, evolutionary history, and natural history of arthropods---the collective focus of the field of ``insect systematics''---represent core knowledge for any entomologist. Whether you are scouting a field, answering a client's questions, prospecting for new research systems, or taking your graduate program's qualifying exam, you will undoubtedly need to recall an insect's name and connect it to some aspect of its biology. Will this insect sting me? Do I need to treat my orchard? When and how did wings evolve? What \textit{is} this thing?

This lab manual was assembled from my experiences in insect systematics courses, both as a student and an instructor. The activities described herein are designed to be done in an immersive lab setting, with a robust teaching collection and all the equipment and supplies one needs to observe and manipulate small organisms (see below). The taxonomic focus is arthropods of the mid-Atlantic and the northeastern USA, but the principles are broadly applicable. The manual works best when complemented by a more comprehensive textbook on insect diversity.

The manual opens with sections focused on foundational knowledge: how to recognize arthropods, what we know about their fossil history, and the methods we use to make sense of this extraordinary diversity---\textit{i.e.}, the fundamentals of systematics. The sections that follow focus on the major arthropod taxa, including all the hexapod orders and select lower-level taxa, usually families or superfamilies, that represent the diversity of life histories in each order. \textbf{Note that this manual is \textit{not} intended as a resource to help you identify all kinds of insects!} Many taxa that are frequently encountered in the eastern USA are not covered at all by this manual. 

\section*{Recommended lab kit}
Before starting the course, each student should invest in the following supplies:
\begin{itemize}
    \item broad-tipped soft forceps (Fisher \#S72114; Fine Science Tools \#26030-10)
    \item narrow-tipped soft forceps (Fisher \#S72113; Fine Science Tools \#26029-10)
    \item \#5 fine-tipped hard forceps (Fine Science Tools \#11254-20)
    \item probe (homemade, using dowel or chopstick and insect pin or minuten pin)
\end{itemize}

\noindent{}Other supplies will likely be provided by the institution, including:
\begin{itemize}
    \item $\sim$80\% ethanol and any other reagents needed for lab
    \item watch glasses
    \item microscopes
    \item any personal protective equipment (PPE) that is deemed necessary (\textit{e.g.}, gloves)
\end{itemize}

\section*{Safety}
We will be working with sharp tools in some of these labs. Wear your personal protective gear at all times. Specimens are to be returned to their vials or trays after lab, and ethanol will be collected for proper disposal or reuse. No food or drink is allowed in the lab space.

\section*{Etiquette}
Teaching collections are complex resources that require extensive curating and re-collecting. Please treat the specimens with \textit{extraordinary} care. Insects are fragile! Please also help maintain the organization and integrity of this resource by putting specimens back into their correct containers and reporting on any breakage that occurs. When you are finished with lab, run through this checklist:
\begin{todolist}
  \item Specimens are put away in their appropriate containers
  \item My microscope is ready to be put away, with the magnification on the lowest setting, the eyepieces and arm lowered all the way, the cord wrapped around the base, and the cover on top
  \item The bench has been wiped down
  \item I looked around for other students that needed help
  \item I checked in with the TA or instructor before leaving
\end{todolist}

\section*{Methods}
Before each lab session, work with a partner to organize your space, specimens, tools, and microscope (see below). Be mindful of the hazards, including sharp instruments and flammable liquids, and do not hesitate to ask questions. You can usually start anywhere within a particular chapter.

\clearpage
\section*{How to set up the microscope}
(We will write this down together, in lab.)