\documentclass[letterpaper, 11pt]{article}
\usepackage{comment} % enables the use of multi-line comments (\ifx \fi) 
\usepackage{fullpage} % changes the margin
\usepackage{fancyhdr} % for footerhttps://www.overleaf.com/3039851kkndbp#
\usepackage[UKenglish]{isodate}% http://ctan.org/pkg/isodate for date format
\usepackage{float}%force tables/figs into certain placement
\usepackage{changepage}%for dichotomous key
\usepackage{graphicx}%for figures
\usepackage{caption}%for figures
\usepackage{subcaption}%for figures
\usepackage{hyperref}%for hyperlinks
\usepackage[font=small,labelfont=bf]{caption}%for captions
\usepackage{natbib}	%for bibliography
\usepackage{placeins}%prevent images from floating into inappropriate sections
\newcommand*{\doi}[1]{\href{http://dx.doi.org/#1}{doi: #1}}% links DOI
\usepackage{epigrafica}%changes default font to epigrafica
\DeclareTextAccent{\"}{OT1}{168}%declare umlaut
\DeclareTextAccent{\~}{OT1}{152}
\usepackage[utf8]{inputenc}%for diacritic on n

\newcommand{\latinword}[1]{\texttt{\itshape #1}}%use \latinword for vocab
\def\labelitemi{--} %change bullet to em dash
\pagestyle{fancy}
\renewcommand{\headrulewidth}{0pt}

\lhead{}
\chead{}
\rhead{}
\lfoot{ENT 432 (Fall 2016) - Penn State}
\cfoot{}
\rfoot{\thepage}
\renewcommand{\footrulewidth}{0.4pt}
\title{Unit 4 - Early arthropods, fossils, terrestrialization}
\author{Open Entomology Project}

\begin{document}
\cleanlookdateon %removed ordinal date
\maketitle
\thispagestyle{fancy}

\section*{Introduction}
A major reason for the incredible diversity of Arthropoda is their successful colonization of terrestrial environments and the ability of these organisms, through the course of evolution, to adapt to the environmental extremes found there. We will discuss hypotheses and evidence concerning the emergence of arthropods from marine habitats. What did early arthropods look like? How are fossils formed, and what kinds of information can we derive from them (and what \textit{can't} we get from them)? Where do we find fossils? In this unit we will address these questions and discuss the challenges faced by arthropods in terrestrial habitats.

\section*{Materials}

\begin{itemize}
\item specimens (provided)
\item fine forceps, probes (provided)
\item sorting tray, watch glasses, gloves, safety glasses, glycerol, ethanol (provided)
\item pencil/lab notebook for sketches
\end{itemize}

\section*{Safety}
We will be working with sharp tools. Wear your personal protective gear at all times. Specimens are to be returned to their vials after lab, and glycerol and ethanol will be collected for proper disposal or reuse.

\section*{Methods}
Working with a partner, organize your space, specimens, tools, and microscope. Use your probe and forceps to manipulate the specimen, unless you're looking at fossils, and don't be afraid to pull parts off or otherwise dissect the specimens if appropriate (not the fossils though!).

\section{Taphonomy}
\latinword{Taphonomy} is the study of how organisms decay after dying, how they settle into positions that may result in fossilization, and how those fossils change chemically, geologically, \textit{etc}. over time. In this lab we have demonstrations that represent two elements of taphonomy: \latinword{decomposition}, the decay of a carcass, and \latinword{biostratinomy}, the processes that affect a carcass prior to deposition. The intention is to get you thinking about biases in the fossil record. The geologic elements of taphonomy, for example, diagenesis, are beyond the scope of this class.

\subsection{Insects trapped in resin}
Amber is fossilized tree resin---\textit{i.e.}, the viscous compounds produced by trees and shrubs (primarily), in response to damage. One can imagine that oozing, sticky goo from tree wounds would be a prime attractant for insects. Many of these insects get trapped, and, if the conditions are right, their fates are preserved for us to see, millions of years later. We'll look at some real amber fossils, but first we'll try an experiment.\\

\hangindent2em\textbf{Question 4-1:} Before moving on to the demonstration jot down a few biases you would expect to see in amber fossils. Think about the insect morphology and natural history. Are there certain behaviors or structures that lend themselves to entrapment or escape? \\

\noindent{}We have several insect specimens and parts of specimens, including an ant, a beetle, and a moth (or moth wing). Using your forceps, try sticking these specimens, one at a time, in a few drops of honey (which simulates resin). Try pouring honey over them. Does it stick?\\

\hangindent2em\textbf{Question 4-2:} What difference---if any---do you see in the way that specimens interact with this sticky substrate. Given these observations, what would you expect to see in the fossil record?

\subsection{Insects in water}
Most non-amber fossils (compression fossils, concretions, ichnofossils, \textit{etc}.) are derived from insects that became trapped or otherwise interacted with aquatic and/or sedimentary environments. Several specimens have been placed in situations that simulate early taphonomy of these fossils. Take some time to observe their condition before answering the next question.\\

\hangindent2em\textbf{Question 4-3:} Describe what you see in each jar. Based on their decomposition, what would you predict about the fossil record and potential biases?

\section{Fossil arthropods}
We discussed arthropod fossils and taphonomy in lecture (see Unit 4 slides and above). Our collection holds examples of three types of fossils that you can examine: 

\paragraph{Creede Formation (Oligocene, $\sim$27 Mya)} You will see several grayish, flat rocks that have compression fossils in them. These insects lived in and around a shallow, high altitude lake in what is now Creede, CO. The lake was repeatedly filled with volcanic ash, from eruptions and from runoff \citep{berkeley}.

\paragraph{Dominican amber (Middle Miocene, 15--20 Mya)} We have a small collection of amber with insect inclusions. These insects lived in what is now the Dominican Republic. This amber is the fossilized resin from a leguminous tree, \textit{Hymenaea protera} (Fabaceae), now extinct.

\paragraph{Barstow Formation (Early to Middle Miocene, 19.3--13.4 Mya)} These fossils are quite small and are mounted on slides. These insects are presumed to have lived in a series of saline-alkaline lakes that was repeatedly subjected to volcanic disturbance. Eruptions repeatedly killed the lakes' fauna and covered the carcasses with ash and calcium carbonate. The subsequent chemistry allowed silica to replace the integument and for the preservation of soft tissues.\\

\hangindent2em\textbf{Question 4-4:} Spend some time with specimens from each type of fossil. Are they biased towards some particular life history or body part, as you hypothesized in section 1? Sketch a couple specimens in your lab notebook and label the parts. Which fossils offer the most information and why?

\section{Terrestrial adaptations}
We also discussed the terrestrialization of Arthropoda and Earth's early history. Take a few minutes to remind yourself of the challenges faced by marine organisms as they moved onto land, before moving on to the next two demonstrations.

\subsection{Respiration}
In Unit 2 you learned about spiracles and tracheae, the primary mechanism through which most terrestrial hexapods respire. Think about how these structures are adapted for an environment that threatens desiccation. Now examine a marine decapod and a xiphosuran (horseshoe crab) and see if you can identify the structures associated with respiration. Can you find analogous structures on a scorpion or a spider?\\

\hangindent2em\textbf{Question 4-5:} Sketch the respiratory structures you can find in the marine decapod, the xiphosuran, and the scorpion. How do the terrestrial taxa deal remediate water loss? 

\subsection{Integument}
Let's take a quick look at some other adaptions to terrestrial, and in this case very, very dry habitats. Death-feigning beetles (Tenebrionidae: \textit{Asbolus verrucosus}) live in the Sonoran Desert. Pick one up and look at it under the microscope. Try to crush it between your fore finger and thumb.\\

\hangindent2em\textbf{Question 4-6:} Did you observe any traits you would attribute to adaptions that prevent water loss?


\section*{Test yourself}
Before moving on to the next unit make sure you're comfortable with the following concepts. Could you write a couple sentences that explain each term? Can you provide examples?

\begin{enumerate} 
\item{Lagerst{\"a}tte} 
\item{taphonomy}  
\item {stem \textit{vs}. crown group}
\end{enumerate}

\noindent{}Can you describe three types of fossils and how they occur? (We discussed five.) What factors influence preservation, and what environmental conditions result in well-preserved fossils? \citep[Hint: see][]{taphonomy2004}\\

\noindent{Given what we know about insect bodies and biology, what biases would you expect to see in the fossil record and why?}\\

\noindent{What can fossils tell us about insect evolution and biology?}\\

\noindent{When did arthropods originate and in what environmental context? What about Insecta?}\\

\noindent{We discussed the evolution of Arthropods through nine geologic periods, beginning with the Cambrian. Can you name these periods and describe in one to three sentences the relevance of each---environmental conditions (generally), types of insects present, and where to find fossils?}\\

\noindent{What characteristics of the arthropod \textit{Bauplan} allowed these organisms to move into terrestrial environments? When and how many times (minimally) did this happen?\\

\section*{Epilogue}
This handout is part of an open curriculum. Original files are available free for anyone to download, copy, modify, and improve at the Open Entomology GitHub repository \citep{ENT432}, which also provides a mechanism for reporting problems and other feedback:\\
\url{https://github.com/OpenEntomology/InsectBiodiversityEvolution/issues}

\FloatBarrier
% adding bibliography here
\bibliographystyle{myplainnat}
\bibliography{bib}
\end{document}
